\documentclass{book}

% Language setting
% Replace `english' with e.g. `spanish' to change the document language
\usepackage[english]{babel}

% Set page size and margins
% Replace `letterpaper' with`a4paper' for UK/EU standard size
\usepackage[letterpaper,top=2cm,bottom=2cm,left=3cm,right=3cm,marginparwidth=1.75cm]{geometry}

% Useful packages
\usepackage{amsmath}
\usepackage{amssymb} % triangle
\usepackage{romannum} % roman numbers
\usepackage{physics} % second partial differential
\usepackage{graphicx}
\usepackage{xcolor}
\usepackage[colorlinks=true, allcolors=blue]{hyperref}
\usepackage{bm}
\setlength\parindent{0pt}

% math typesetting
\newcommand*{\vertbar}{\rule[-1ex]{0.5pt}{2.5ex}}
\newcommand*{\horzbar}{\rule[.5ex]{2.5ex}{0.5pt}}
\usepackage{scalerel,stackengine,amsmath}
\newcommand\equalhat{\mathrel{\stackon[1.5pt]{=}{\stretchto{%
    \scalerel*[\widthof{=}]{\wedge}{\rule{1ex}{3ex}}}{0.5ex}}}}

\includeonly{la-basics,pca}


\title{Basic methods for data analysis}
\author{Philipp Renz}
\begin{document}
\frontmatter
\maketitle
\tableofcontents

\mainmatter
\chapter{Introduction}
\chapter{Linear algebra: Basics}

In data analysis we will often represent our data as matrices. Consequently many algorithms are described using
the concept of matrices and their properties which could be broadly called the study of linear algebra. Here I
don’t want to give an axiomatic introduction but rather provide some basic results with the prerequisite that
the reader already knows some basics about vectors and matrices.
We will denote vectors as bold lowercase letters 
\begin{equation}
\bm{a} = 
\begin{pmatrix}
a_1     \\
\vdots  \\
a_n
\end{pmatrix},
\end{equation}
where $a_i$ is the $\text{i}^{th}$ entry of the vector $\bm{a}$.

Matrices are denoted by bold uppercase letters
\begin{equation}
    \bm{A} = 
    \begin{pmatrix}
        a_{11} & \hdots & a_{1n} \\ 
        \vdots & \ddots & \vdots \\
        a_{m1} & \hdots & a_{mn}
    \end{pmatrix}.
\end{equation}
Depending on the context it is useful to view a matrix as a concetation of either row or column vectors.
\begin{align}
    \bm{A} = \begin{pmatrix}
        \horzbar & \bm{a}_1^r & \horzbar \\
                  & \vdots   &  \\
        \horzbar & \bm{a}_m^r &  \horzbar \\
    \end{pmatrix} = 
    \begin{pmatrix}
    \vertbar &        & \vertbar \\
    \bm{a}_1^c & \hdots & \bm{a}_n^c \\
    \vertbar &        & \vertbar 
    \end{pmatrix}.
\end{align}
We made an explicit distinction between row- and column vectors using superscripts here but these will often be left out
for better readability.

\section{Matrix multiplication}
Given two matrices 
\begin{align}
    \bm{A} = 
    \begin{pmatrix}
        a_{11} & \hdots & a_{1n} \\ 
        \vdots & \ddots & \vdots \\
        a_{m1} & \hdots & a_{mn}
    \end{pmatrix} \in R^{m \times n}, &&
    \bm{B} = 
    \begin{pmatrix}
        b_{11} & \hdots & b_{1p} \\ 
        \vdots & \ddots & \vdots \\
        b_{n1} & \hdots & b_{np}
    \end{pmatrix} \in R^{n \times p}
\end{align}
their product $\bm{C} = \bm{A}\bm{B}$ is of shape $(m \times p)$ and it's entries are
\begin{equation}
    c_{ik} = \sum_{j=1}^n a_{ij} b_{jk}
\end{equation}

\newcommand{\x}{\times}
The resulting matrix $\bm{C}$ has a shape of $(m \x p) \leftarrow (m \x n)(n \x p)$. The inner dimension $n$ is "eliminated" by summing over it. One can also see that the sum moves horizontally/rowwise over the entries of A and down the
columns of B

Often we are interested in a special case in which one of the matrices reduces to a vector, that means it has only
one column or one row. We define a column vector as a shorthand notation for a matrix $\bm{B} \in R^{n \x 1}$ with only one column and drop the column index:

\begin{equation}
    \bm{B} =
    \begin{pmatrix}
    b_{11} \\
    \vdots \\
    b_{n1}
    \end{pmatrix}     \equalhat
    \begin{pmatrix}
        b_{1} \\
        \vdots \\
        b_{n}
    \end{pmatrix} = \bm{b}
\end{equation}

Implicitly when we talk of a vector b with n elements we mean a matrix $\bm{B} \in R^{n\x 1}$. When we want to talk
about a matrix with only one row we write this a transposed vector
\begin{equation}
    \bm{b^T} = (b_1 \hdots b_n) \equalhat (b_{11} \hdots b_{1n})
\end{equation}

\subsection{Vector-vector products}
Two vectors can be multiplied with each other in two different ways. 
The scalar or inner product is defined as 
\begin{equation}
    \bm{a} \cdot \bm{b} = \bm{a}^T \bm{b} = \sum_{i=1}^n a_i b_i 
\end{equation}
This can be seen as a multiplication of matrices with shapes $(1 \x m)(m \x 1) \rightarrow (1 \x 1)$ and results in a scalar value.

The outer product between two vectors is 
\begin{equation}
    \bm{C} = \bm{a}\bm{b}^T = \begin{pmatrix}
    a_{1} \\
    \vdots \\
    a_{m}
    \end{pmatrix}
    \begin{pmatrix}
        b_1 & \hdots & b_n
    \end{pmatrix} = 
    \begin{pmatrix}
        a_1 b_1 & a_1 b_2 & \hdots & a_1 b_n \\
        a_2 b_1 & a_2 b_2 & \hdots & a_2 b_n \\
        \vdots & \vdots   & \ddots & \vdots  \\
        a_m b_1 & a_m b_2 & \hdots & a_m b_n
    \end{pmatrix}
\end{equation}

This means that every column is a multiple of $\bm{a}$ and every row a multiple of $\bm{b}$. 
We see that the entry $c_{ij} = a_i b_j$.

\subsection{Matrix-vector multiplication}
The product of a matrix and vector $\bm{c} = \bm{A}\bm{b}$ again is a vector. 
If $\bm{A} \in R^{m \x n}$ and $\bm{b} \in R^n$ their product
will be of shape $(m \x 1)$, or $\bm{c} \in R^m$.

\subsubsection{Linear combination of columns}

One can view matrix-vector product, $\bm{c} = \bm{Ab}$, as calculating a linear combination of the columns of $\bm{A}$ weighted by 
the entries of $\bm{b}$:


\begin{align}
\bm{Ab} &= \begin{pmatrix}
    \sum_{j=1}^n a_{1j}b_j\\
    \sum_{j=1}^n a_{2j}b_j\\
    \vdots \\
    \sum_{j=1}^n a_{mj}b_j\\
\end{pmatrix} = 
{\setlength\arraycolsep{1pt}
\begin{pmatrix}
    \red{a_{11}} b_1 &+& \blue{a_{12}} b_2 &+&\hdots &+& \green{a_{1n}} b_n \\
    \red{a_{21}} b_1 &+& \blue{a_{22}} b_2 &+&\hdots &+& \green{a_{2n}} b_n \\
    \red{\vdots} \ \ & & \blue{\vdots} \ \ & &\ddots & & \green{\vdots} \ \ \\
    \red{a_{m1}} b_1 &+& \blue{a_{m2}} b_2 &+&\hdots &+& \green{a_{mn}} b_n \\
\end{pmatrix}} = 
{\setlength\arraycolsep{1pt}
\begin{pmatrix}
\red{\vertbar} \ \ & & \blue{\vertbar} \ \ & &        & & \green{\vertbar} \ \ \\
\red{\bm{a}_1} b_1 &+& \blue{\bm{a}_2} b_2 &+& \hdots &+& \green{\bm{a}_n} b_n \\
\red{\vertbar} \ \ & & \blue{\vertbar} \ \ & &        & & \green{\vertbar} \ \
\end{pmatrix}}  \\
&=\red{\bm{a}_1} b_1 +  \blue{\bm{a}_2} b_2 + \hdots + \green{\bm{a}_n} b_n = \sum_{i=1}^n \bm{a}_i b_i
\end{align}

One prominent example where this view is useful is that if the columns of $\bm{A}$ form a basis.
Then a vector $\bm{c}$ can be written as $\bm{c} = \bm{A}\bm{c}_A$, where 
$\bm{c}_A$ are the coordinates of $\bm{c}$ with respect to the basis $\bm{A}$.
Equivalently if the columns of $\bm{B}$ form another basis then
\begin{equation}
    \bm{c} = \bm{A}\bm{c}_A = \bm{B}\bm{c}_B.
\end{equation}
From this equation one can easily get the prescription of calculating
the coordinates wrt. to $\bm{A}$ given those wrt. $\bm{B}$,

\begin{equation}
    \bm{c}_A = \bm{A}^{-1} \bm{B}\bm{c}_B.
\end{equation}

\subsubsection{Dot product}
Alternatively we can view a matrix vector product $\bm{Ab}$ as 
calculating the dot products of $\bm{b}$ with each row of $\bm{A}$,
\begin{equation}
    \bm{Ab} = \begin{pmatrix}
        \horzbar & \bm{a}_1 & \horzbar \\
                  & \vdots   &  \\
        \horzbar & \bm{a}_m &  \horzbar \\
    \end{pmatrix} 
    \begin{pmatrix}
    \vertbar \\
    \bm{b}  \\
    \vertbar
    \end{pmatrix}= 
    \begin{pmatrix}
    \bm{a}_{1} \cdot \bm{b} \\
    \vdots \\
    \bm{a}_{m} \cdot \bm{b} \\
    \end{pmatrix}
\end{equation}

\subsection{Matrix-matrix multiplication}
Now that we've looked at some special cases of matrix multiplication
we can go back to the general case 
$\bm{C} = \bm{AB}$.

Recalling the definition of matrix multiplication 
\begin{equation}
    c_{ik} = \sum_{j=1}^n a_{ij} b_{jk}
\end{equation}
we can look at the index $k$ and see that the $k$-th column of $\bm{C}$ is formed by the matrix vector product of $A$ and 
the $k$-th column vector of $\bm{B}$,
\begin{equation}
    \bm{C} = \bm{AB} = \bm{A} \begin{pmatrix}
    \vertbar &        & \vertbar \\
    \bm{b}_1 & \hdots & \bm{b}_p \\
    \vertbar &        & \vertbar 
    \end{pmatrix} = 
    \begin{pmatrix}
        \vertbar &        & \vertbar \\
        \bm{A}\bm{b}_1 & \hdots & \bm{A}\bm{b}_p \\
        \vertbar &        & \vertbar 
    \end{pmatrix} = 
\begin{pmatrix}
\vertbar &        & \vertbar \\
\bm{c}_1 & \hdots & \bm{c}_p \\
\vertbar &        & \vertbar 
\end{pmatrix}.
\end{equation}
Breaking matrix-matrix multiplication down to multiple matrix vector operations
allows us to view it as calculating multiple linear combinations of the columns of $\bm{A}$
or calculating the all inner products between the rows of $\bm{A}$ and the columns of $\bm{B}$,
\begin{equation}
    \bm{C} = \begin{pmatrix}
        \horzbar & \bm{a}_1 & \horzbar \\
                  & \vdots   &  \\
        \horzbar & \bm{a}_m &  \horzbar \\
    \end{pmatrix} 
    \begin{pmatrix}
    \vertbar &        & \vertbar \\
    \bm{b}_1 & \hdots & \bm{b}_p \\
    \vertbar &        & \vertbar 
    \end{pmatrix} = 
    \begin{pmatrix}
        \bm{a}_{1} \cdot \bm{b}_1 & \bm{a}_{1} \cdot \bm{b}_2 & \hdots & \bm{a}_{1} \cdot \bm{b}_p  \\
        \bm{a}_{2} \cdot \bm{b}_1 & \bm{a}_{2} \cdot \bm{b}_2 & \hdots & \bm{a}_{2} \cdot \bm{b}_p  \\
        \vdots                    & \vdots                    & \ddots & \vdots                     \\
        \bm{a}_{m} \cdot \bm{b}_1 & \bm{a}_{m} \cdot \bm{b}_2 & \hdots & \bm{a}_{m} \cdot \bm{b}_p  \\
    \end{pmatrix}.
\end{equation}
From this we can read that $c_{ik} = \bm{a}_i \bm{b}_j$.

\subsubsection{Sum of outer products}
Another view that is often useful is to 
write a matrix-matrix multiplication as a sum of outer products.
Looking again at our basic definition
\begin{equation}
    c_{ik} = \sum_{j=1}^n a_{ij} b_{jk} = a_{i1} b_{1k} + a_{i2} b_{2k} + a_{i3} b_{3k}
\end{equation}
we can see that for example the term $a_{i2} b_{2k}$ is the $ik$-th entry of the outer-product between
the second column vector of $\bm{A}$ and the second row vector of $\bm{B}$.
Thus we can deduce that the matrix $\bm{C}$ is a sum of outer-products,
\begin{equation}
    \bm{C} = \bm{AB} = \begin{pmatrix}
    \vertbar &        & \vertbar \\
    \bm{a}_1 & \hdots & \bm{a}_n \\
    \vertbar &        & \vertbar 
    \end{pmatrix}
    \begin{pmatrix}
        \horzbar & \bm{b}_1 & \horzbar \\
                  & \vdots   &  \\
        \horzbar & \bm{b}_n &  \horzbar \\
    \end{pmatrix}  = 
    \sum_{i=1}^n \bm{a}_i \bm{b}_i^T.
\end{equation}

\section{Orthogonal matrices}

An orthogonal matrix is a square matrix with the special property that its transposed is equal to its inverse.
That is for $\bm{U} \in \mathbb{R}^{m \x m}$
\begin{align}
    \bm{U^T} \bm{U} &= \bm{U} \bm{U^T} = \bm{1} \\
    \bm{U}^T &= \bm{U}^{-1}
\end{align}

Viewing this matrix-matrix product as pairwise dot-products between the columns/rows of the two matrices
reveals that the both the columns as well as the rows of $\bm{U}$ form an orthonormal set of vectors:
\begin{align}
    \begin{pmatrix}
        \horzbar & \bm{u}_1 & \horzbar \\
                  & \vdots   &  \\
        \horzbar & \bm{u}_m &  \horzbar \\
    \end{pmatrix} 
    \begin{pmatrix}
    \vertbar &        & \vertbar \\
    \bm{u}_1 & \hdots & \bm{u}_m \\
    \vertbar &        & \vertbar 
    \end{pmatrix} = 
    \begin{pmatrix}
        \bm{u}_{1} \cdot \bm{u}_{1} & \bm{u}_{1} \cdot \bm{u}_{2} & \hdots & \bm{u}_{1} \cdot \bm{u}_{m} \\
        \bm{u}_{2} \cdot \bm{u}_{1} & \bm{u}_{2} \cdot \bm{u}_{2} & \hdots & \bm{u}_{2} \cdot \bm{u}_{m} \\
        \vdots                    & \vdots                        & \ddots & \vdots                      \\
        \bm{u}_{m} \cdot \bm{u}_{1} & \bm{u}_{m} \cdot \bm{u}_{2} & \hdots & \bm{u}_{m} \cdot \bm{u}_{m} \\
    \end{pmatrix}   
    \begin{pmatrix}
        1 & 0 & \hdots & 0 \\
        0 & 1 & \hdots & 0 \\
        \vdots & \vdots & \ddots & \vdots \\
        0 & 0 & \hdots & 1
    \end{pmatrix}
\end{align}

The transformation can be seen as a rotation followed by reflection.
This is because angles and lengths are preserved under the transformation.
To see this take any two vectors $\bm{a}, \bm{b}$ and apply the transform $\bm{U}$.
We first establish that multiplication of a vector with $\bm{U}$ does not change its length:
\begin{align}
    \norm{\bm{Ua}}^2 = \bm{a}^T \red{\bm{U}^T\bm{U}} \bm{a} =\bm{a^T} \red{\bm{1}} \bm{a} = \norm{\bm{a}}^2
\end{align}

\begin{align}
    \cos{(\angle(\bm{Ua}, \bm{Ub}))} =&   \frac{\bm{U}\bm{a} \cdot \bm{U} \bm{b}}{\norm{\bm{Ua}} \ \norm{\bm{Ub}} }\\
    =& \frac{\bm{a}^T \bm{U^T} \bm{U} \bm{b}}{\norm{\bm{Ua}} \ \norm{\bm{Ub}} }\\
     =& \frac{\bm{a}^T \bm{b}}{\norm{\bm{a}} \norm{\bm{b}}} = \cos{(\angle(\bm{a}, \bm{b}))}.
\end{align}
\newcommand{\R}{\mathbb{R}}
\section{Eigenvalues and eigenvectors}
A square matrix $\bm{A} \in \R^{n \x n}$ can be seen as a linear map that maps a vector $\bm{v} \in \R^n$ to a vector $\bm{Av} \in \R^n$.
There are some vectors called \emph{eigenvectors} (derived from the german word "eigen") which are mapped onto a multiple of themselves,
\begin{align}
    \bm{Av} = \lambda \bm{v}.
    \label{eq:evdef}
\end{align}
If this equation holds $\bm{v}$ is called an eigenvector and $\lambda$ its associated eigenvalue.
The eigenvalues are a useful characteristic of a matrix and finding eigenvalues has lots of applications ranging from coupled oscillators, heat and Schrödinger equations in
physics to principal component analysis in data analysis.

We still need a way to find the eigenvalues/vectors. 
From \eqref{eq:evdef} we get 

\begin{align}
    \bm{Av} =& \ \lambda \bm{1}\bm{v} \\
    (\bm{A} - \lambda \bm{1})\bm{v} =& \ \bm{0}
\end{align}

This yields a linear system of equations, which only has a solution if the columns of 
$ \bm{A}-\lambda \bm{1}$ are linearly dependent.
Thus we can find eigenvalues by setting the determinant of this matrix to zero
\begin{align}
    \det(\bm{A} - \lambda \bm{1}) = 0    
\end{align}
This gives a polynomial of degree n, called the characteristic polynomial, which according to the fundamental
theorem of calculus has $n$ roots and can be written as
\begin{align}
    \det(\bm{A} - \lambda \bm{1}) = \prod_{i=1}^n (\lambda - \lambda_i),
\end{align}
where the $\lambda_i$ are solutions to the equation above. 
The number of times a specific value occurs in this factorization is called 
an eigenvalues algebraic multiplicity.
After calculating the eigenvalues one can insert them into \eqref{eq:??} one ofter another and solve for the eigenvectors
$v_i$. One interesting question is whether the set of all eigenvectors form a complete basis of $\R^n$. 
This is not the case if for an eigenvalue with algebraic multiplicity $n_a > 1$ the null space of $\bm{A} - \lambda_i$ has dimension less than $n_a$.


\paragraph{Example} Given a matrix
\begin{align}
\bm{A} = 
\begin{pmatrix}
    1 & 1 \\
    0 & 1
\end{pmatrix}
\end{align}
We can calculate it’s characteristic polynomial
\begin{align}
    \det(\bm{A} - \lambda \bm{1}) = (1-\lambda)^2
\end{align}
Thus we get an eigenvalue $\lambda = 1$ with algebraic multiplicity of two. However when inserting this eigenvalue to
solve for $\bm{v}$ we get
\begin{align}
    (\bm{A} - \lambda \bm{1})  \bm{v} =  \begin{pmatrix}
        0 & 1 \\
        0 & 0
    \end{pmatrix}
    \begin{pmatrix}
        v_1 \\
        v_2
    \end{pmatrix}
    \Rightarrow \bm{v} = s \begin{pmatrix}
        1 \\
        0
    \end{pmatrix}
\end{align}

\subsection{Diagonalization of a matrix}
If we find an eigenbasis, that means a basis that only consists of eigenvectors, we can write the following:
\begin{align}
    \bm{A} \begin{pmatrix}
    \vertbar &        & \vertbar \\
    \bm{v}_1 & \hdots & \bm{v}_n \\
    \vertbar &        & \vertbar 
    \end{pmatrix} = 
    \begin{pmatrix}
        \vertbar &        & \vertbar \\
        \lambda_1\bm{v}_1 & \hdots & \lambda_n\bm{v}_n \\
        \vertbar &        & \vertbar 
    \end{pmatrix} = 
    \begin{pmatrix}
        \vertbar &        & \vertbar \\
        \bm{v}_1 & \hdots & \bm{v}_n \\
        \vertbar &        & \vertbar 
    \end{pmatrix}
    \begin{pmatrix}
        \lambda_1 &        &           \\
                  & \ddots &           \\
                  &        & \lambda_n
        \end{pmatrix}
\end{align} or
\begin{align}
    \bm{AV} = \bm{V\Lambda}
\end{align}
Since the columns of $\bm{V}$ form a basis they are linear independent and one can invert $\bm{V}$. Thus if there is an
eigenbasis a matrix can be ”diagonalized”:
\begin{align}
    \bm{A} = \bm{V\Lambda V}^{-1} && \bm{\Lambda} = \bm{V}^{-1} \bm{\Lambda V}^{-1}
\end{align}

This form is often practical, for example it’s easy to compute
\begin{align}
    \bm{A}^p &=  \underbrace{\bm{V\Lambda V}^{-1} \bm{V\Lambda V}^{-1} \hdots \bm{V\Lambda V}^{-1}}_{p \ \text{times}} \\
             &=  \bm{V\Lambda}^p \bm{V}^{-1} = \bm{V} 
                \begin{pmatrix}
                    \lambda_1^p &        &             \\
                                & \ddots &             \\
                                &        & \lambda_n^p \\
                \end{pmatrix} \bm{V}^{-1}.
\end{align}


\section{Definiteness}
A matrix $\bm{A}$ is (strictly) positive-definite if 
\begin{align}
    \bm{v}^T \bm{Av} > 0 \ \quad \forall\quad \bm{v} \neq 0
\end{align}
Similarly $\bm{A}$ is positive semi-definite if 
\begin{align}
    \bm{v}^T \bm{Av} \geq 0 \ \quad \forall\quad \bm{v} \neq 0.
\end{align}

A matrix $\bm{A}$ is (strictly) negative-definite if 
\begin{align}
    \bm{v}^T \bm{Av} < 0 \ \quad \forall\quad \bm{v} \neq 0.
\end{align}

A matrix $\bm{A}$ is negative-semidefinite if 
\begin{align}
    \bm{v}^T \bm{Av} \leq 0 \ \quad \forall\quad \bm{v} \neq 0.
\end{align}

\paragraph{Note:} Don't be confused if definitions vary. Some people say positive-definite for non-strict inequalities and then 
specify the strict inequality using "stricly positive-definite". Some specifically say positive-semidefinite if they mean non-strict inequality.
Often the meaning is apparent from the context.

\section{Symmetric matrices}
A matrix $\bm{A}$ is symmetric if $\bm{A} = \bm{A}^T$ that is if $a_{ij} = a_{ji}$.
Eigenvalues of real symmetric matrices are real.
Suppose $\lambda$ is a possibly complex eigenvalue of symmetric matrix $\bm{A}$ and eigenvector $\bm{v} \neq 0$.

\paragraph{Proof:} 
\begin{align}
      &\lambda {\bar{\bm{v}}}^T \bm{v}   = \bar{\bm{v}}^T \lambda \bm{v} = \bar{\bm{v}}^T A\bm{v} = \bar{\bm{v}}^T A^T \bm{v}  =         \\
     =&(A\bar{\bm{v}})^T \bm{v}       = (\bar{A}\bar{\bm{v}})^T \bm{v} = (\bar{\lambda}\bar{\bm{v}})^T = \bar{\lambda}\bar{\bm{v}}^T \bm{v}  
\end{align}
From this we see that $\lambda = \bar{\lambda}$, where we used the fact that $\overline{\bm{Av}} = \bm{\bar{A}}$    

For real symmetric matrices eigenvectors with differing related eigenvalues are orthogonal.
To see this consider the following calculation:
\begin{align}
    \lambda_{1} \bm{v}_1^T \bm{v}_2 = \bm{v}_1^T A^T \bm{v}_2 = \bm{v}_1^T A \bm{v}_2 = \bm{v}_1^T \bm{v}_2 \lambda_{2}.
\end{align}
Here we see that if $\lambda_{1} \neq \lambda_{2}$ it follows that $\bm{v}_1 \cdot \bm{v}_2 = 0$.

If an eigenvalue is associated with multiple eigenvectors the eigenvectors span a higher dimensional eigenspace. For example, if both 
$\bm{v}_1$ and $\bm{v}_2$ have an eigenvalue of $\lambda$ we get that
\begin{align}
    \bm{A}(a\bm{v}_1+b\bm{v}_2)=a\bm{A}\bm{v}_1+ b \bm{A}\bm{v}_2=a\lambda\bm{v}_1+b\lambda\bm{v}_2= \lambda(a\bm{v}_1+b\bm{v}_2).
\end{align}
Thus every linear combination of the two vectors is also an eigenvector.    
Use gram-schmidt to get orthogonal basis for this space.

Each symmetric real matrix has an orthogonal eigenbasis. 
In case all eigenvalues are different from each other this follows from above.
The general case is a bit more complicated but we'll give an intuition below.
To do this we first need to establish that we can always find at least one non-trivial eigenvector 
for a matrix $\bm{A}$. This is because the characteristic polynomial will have at least one solution.
Thus we can find an eigenvector $\bm{v}_1$. Then we take any vector $\bm{u}$ that is orthogonal to $\bm{v}_1$.
We observe that 
\begin{align}
   0 = \bm{u}^T\bm{v}_1 \lambda_1 = \bm{u}^T\bm{Av}_1  = \bm{u}^T\bm{A}^T\bm{v}_1  = (\bm{A}\bm{u})^T\bm{v}_1
\end{align}
Thus we know that $\bm{Au}$ stays orthogonal to $\bm{v_1}$.
This means that we can define a new subspace $V_1^\top$ that is orthogonal to $\bm{v}_1$, i.e. it only contains vectors orthogonal 
to $\bm{v}_1$. 
Now we know that $\bm{A}$ maps all vectors $V_1^\top$ to the same space.
This means we can again find an eigenvector for this mapping. 
We could in principle transform the linear mapping that $\bm{A}$ represents into a basis of $V_1^\top$.
Then we would have the familiar matrix form again. 
In this way we would get a second eigenvector $\bm{v}_2$. 
This can be done until one has a complete set of eigenvectors. By construction they will all be orthogonal to each other.



   

    \section{Singular value decomposition}
    The singular value decomposition is a useful way to factorize a matrix. 
    Any matrix $\bm{M} \in R^{m \x n}$ can be written as a product of three matrices
    \begin{align}
        \underset{(m \x n)}{\bm{M}}  = \underset{(m \x m)}{\bm{U}}\underset{(m \x n)}{\bm{\Sigma}} \underset{(n \x n)}{\bm{V}^T}\\,
    \end{align}
    
    where $\bm{U}$ and $\bm{V}$ are orthogonal 
    \begin{align}
        \bm{UU}^T = \bm{1}_{m \x m} && \bm{VV}^T = \bm{1}_{n \x n},
    \end{align}
    and $\bm{\Sigma}$ is diagonal.
    
    Without loss of generality we can assume that $m \geq n$. This would only result in empty columns instead of empty rows of $\bm{\Sigma}$.
    The shapes look approximately like this ($5 \x 3$) in the example.
    \begin{align}
        \begin{bmatrix}
        &        & \\
        &        & \\
        & \bm{M} & \\
        &        & \\
        &        & 
        \end{bmatrix}    = 
        \begin{bmatrix}
        & &        & & \\
        & &        & & \\
        & & \bm{U} & & \\
        & &        & & \\
        & &        & & 
        \end{bmatrix}_{m \x m}
        \begin{bmatrix}
        \sigma_1 &        &          \\
                 & \ddots &          \\
                 &        & \sigma_n \\
                 &        &          \\
                 &        &          
        \end{bmatrix}_{m \x n}
        \begin{bmatrix}
            &          & \\
            & \bm{V}^T & \\
            &          &  
        \end{bmatrix}_{n \x n}
    \end{align}
    
    \paragraph{Proof of existence}
    We will prove that such a representation always exists.
    We know that $\bm{M}^T \bm{M}$ has an orthogonal eigenbasis as it is symmetric, and non-negative eigenvalues as it is positive-definite,
    \begin{align}
        \bm{M}^T \bm{M} \bm{v}_i = \alpha_i \bm{v}_i, && \alpha_i \geq 0, &&  \bm{v}_i \cdot \bm{v}_j = \delta_{ij}.
    \end{align}
    Next we define vectors 
    \begin{align}
    \bm{u}_i = \frac{\bm{Mv}_i}{\alpha_i}    
    \end{align}
    and observe that they are also pairwise orthogonal
    \begin{align}
        \bm{u}_i \bm{u}_j = \frac{\bm{v}_i^T \bm{M}^T \bm{Mv}_i}{\sqrt{\alpha_i \alpha_j}} = \frac{\bm{v}_i^T \bm{v}_i \alpha_i}{{\sqrt{\alpha_i \alpha_j}}} = 
        \frac{\delta_{ij} \alpha_i}{{\sqrt{\alpha_i \alpha_j}}} = 
        \frac{\delta_{ij} \alpha_i}{{\sqrt{\alpha_i \alpha_i}}} = \delta_{ij}
    \end{align}
    
    We also get that the $\bm{u}_i$ are eigenvectors of $\bm{MM}^T$, 
    \begin{align}
         (\bm{M}\bm{M})^T \bm{u}_i
        =(\bm{M}\bm{M})^T \bm{Mv}_i \frac{1}{\sqrt{\alpha_i}} 
        =\bm{M}(\bm{M}^T \bm{M}) \bm{v}_i \frac{1}{\sqrt{\alpha_i}} 
        = \alpha_i \bm{M} \bm{v}_i  \frac{1}{\sqrt{\alpha_i}} 
        = \alpha_i \bm{u}_i
    \end{align}
    
    Next we assume that$\bm{M}^T \bm{M}$ has $l$  non-zero eigenvalues.
    We take the respective eigenvectors $\bm{v}_1, \hdots, \bm{v}_l$  and compute the corresponding $\bm{u}_i$ 
    simultaneously, while using $\sigma_i=\alpha_i^{-\frac{1}{2}}$,
    \begin{align}
        \begin{pmatrix}
        \vertbar &        & \vertbar \\
        \bm{u}_1 & \hdots & \bm{u}_l \\
        \vertbar &        & \vertbar 
        \end{pmatrix} =&
        \bm{M}
        \begin{pmatrix}
        \vertbar &        & \vertbar \\
        \bm{v}_1 & \hdots & \bm{v}_l \\
        \vertbar &        & \vertbar 
        \end{pmatrix}
        \begin{pmatrix}
        \sigma_1 &        &          \\
                 & \ddots &          \\ 
                 &        & \sigma_l
        \end{pmatrix} \\
        \begin{pmatrix}
        \vertbar &        & \vertbar \\
        \bm{u}_1 & \hdots & \bm{u}_l \\
        \vertbar &        & \vertbar 
        \end{pmatrix}
        \begin{pmatrix}
            \sigma_1 &        &          \\
                     & \ddots &          \\ 
                     &        & \sigma_l
            \end{pmatrix}
        =&
        \bm{M}
        \begin{pmatrix}
        \vertbar &        & \vertbar \\
        \bm{v}_1 & \hdots & \bm{v}_l \\
        \vertbar &        & \vertbar 
        \end{pmatrix} \\
        \begin{pmatrix}
        \vertbar &        & \vertbar \\
        \bm{u}_1 & \hdots & \bm{u}_l \\
        \vertbar &        & \vertbar 
        \end{pmatrix}
        \begin{pmatrix}
            \sigma_1 &        &          \\
                     & \ddots &          \\ 
                     &        & \sigma_l
        \end{pmatrix}
        \begin{pmatrix}
            \horzbar & \bm{v}_1 & \horzbar \\
                      & \vdots   &  \\
            \horzbar & \bm{v}_l &  \horzbar \\
        \end{pmatrix} 
        =&
        \bm{M}
        \begin{pmatrix}
        \vertbar &        & \vertbar \\
        \bm{v}_1 & \hdots & \bm{v}_l \\
        \vertbar &        & \vertbar 
        \end{pmatrix}
        \begin{pmatrix}
            \horzbar & \bm{v}_1 & \horzbar \\
                      & \vdots   &  \\
            \horzbar & \bm{v}_l &  \horzbar \\
        \end{pmatrix} 
    \end{align}
    
    To get to our desired factorization and isolate $\bm{M}$ we would like to get somehow get rid of the last to matrices on the right.
    However while $\bm{V}^T \bm{V} = \bm{1}_{l \x l}$, $\bm{V}^T \bm{V} = \bm{1}_{n \x n}$ does not hold in general, and we can't easily 
    cancel it out.
    We know however, that 
    \begin{align}
        \bm{v}_i^T \bm{M}^T \bm{M} \bm{v_i} =  \bm{v}_i^T \bm{v}_i \alpha_i \\
        \alpha_i = 0 \ \Rightarrow \ \bm{Mv}_i = 0
    \end{align}
    \allowdisplaybreaks
    
    This allows us to do the following, where the first equality holds because we use the full set of eigenvectors $n$ eigenvectors,
    \begin{align}
        &\bm{M} = \bm{M}
        \begin{pmatrix}
        \vertbar &        & \vertbar & \red{\vertbar    } & \red{        }& \red{ \vertbar }\\
        \bm{v}_1 & \hdots & \bm{v}_l & \red{\bm{v}_{l+1}} & \red{ \hdots }& \red{ \bm{v}_n }\\
        \vertbar &        & \vertbar & \red{\vertbar    } & \red{        }& \red{ \vertbar }
        \end{pmatrix}
        \begin{pmatrix}
            \horzbar & \bm{v}_1   & \horzbar \\
                     & \vdots     &          \\
            \horzbar & \bm{v}_l   & \horzbar \\
            \red{\horzbar} & \red{\bm{v}_l+1} & \red{\horzbar} \\
                           & \red{\vdots    } &                \\
            \red{\horzbar} & \red{\bm{v}_n  } & \red{\horzbar} \\
        \end{pmatrix} \\
        =&
        \begin{pmatrix}
        \vertbar       &        & \vertbar       & \red{\vertbar    }       & \red{        }& \red{ \vertbar }       \\
        \bm{M}\bm{v}_1 & \hdots & \bm{M}\bm{v}_l & \red{\bm{M}\bm{v}_{l+1}} & \red{ \hdots }& \red{\bm{M} \bm{v}_n } \\
        \vertbar       &        & \vertbar       & \red{\vertbar    }       & \red{        }& \red{ \vertbar }
        \end{pmatrix}
        \begin{pmatrix}
            \horzbar & \bm{v}_1   & \horzbar \\
                     & \vdots     &          \\
            \horzbar & \bm{v}_l   & \horzbar \\
            \red{\horzbar} & \red{\bm{v}_l+1} & \red{\horzbar} \\
                           & \red{\vdots    } &                \\
            \red{\horzbar} & \red{\bm{v}_n  } & \red{\horzbar} \\
        \end{pmatrix} \\
        =&
        \begin{pmatrix}
        \vertbar       &        & \vertbar       & \red{\vertbar} & \red{        } & \red{\vertbar} \\
        \bm{M}\bm{v}_1 & \hdots & \bm{M}\bm{v}_l & \red{\bm{0}  } & \red{ \hdots } & \red{\bm{0}}   \\
        \vertbar       &        & \vertbar       & \red{\vertbar} & \red{        } & \red{\vertbar}
        \end{pmatrix}
        \begin{pmatrix}
            \horzbar & \bm{v}_1   & \horzbar \\
                     & \vdots     &          \\
            \horzbar & \bm{v}_l   & \horzbar \\
            \red{\horzbar} & \red{\bm{v}_l+1} & \red{\horzbar} \\
                           & \red{\vdots    } &                \\
            \red{\horzbar} & \red{\bm{v}_n  } & \red{\horzbar} \\
        \end{pmatrix}\\
        =& \bm{M}\begin{pmatrix}
        \vertbar       &        & \vertbar       & \red{\vertbar} & \red{        } & \red{\vertbar} \\
        \bm{v}_1 & \hdots & \bm{v}_l & \red{\bm{0}  } & \red{ \hdots } & \red{\bm{0}}   \\
        \vertbar       &        & \vertbar       & \red{\vertbar} & \red{        } & \red{\vertbar}
        \end{pmatrix}
        \begin{pmatrix}
            \horzbar & \bm{v}_1   & \horzbar \\
                     & \vdots     &          \\
            \horzbar & \bm{v}_l   & \horzbar \\
            \red{\horzbar} & \red{\bm{v}_l+1} & \red{\horzbar} \\
                           & \red{\vdots    } &                \\
            \red{\horzbar} & \red{\bm{v}_n  } & \red{\horzbar} \\
        \end{pmatrix} \\
        =&
        \bm{M} \sum_{i=1}^l \bm{v}_i \bm{v}_i^T + \sum_{i=l+1}^n \bm{0} \ \bm{v}_i^T =
        \bm{M} \sum_{i=1}^l \bm{v}_i \bm{v}_i^T  \\
        =&     \bm{M}
        \begin{pmatrix}
        \vertbar &        & \vertbar \\
        \bm{v}_1 & \hdots & \bm{v}_l \\
        \vertbar &        & \vertbar 
        \end{pmatrix}
        \begin{pmatrix}
            \horzbar & \bm{v}_1 & \horzbar \\
                      & \vdots   &  \\
            \horzbar & \bm{v}_l &  \horzbar \\
        \end{pmatrix}.
    \end{align}
    This is just the rhs. of \eqref{?} and we get that 
    \begin{align}
        \bm{M} = 
        \underbrace{
            \begin{pmatrix}
                \vertbar &        & \vertbar \\
                \bm{u}_1 & \hdots & \bm{u}_l \\
                \vertbar &        & \vertbar 
            \end{pmatrix}
        }_{\bm{U}^{(1)}}
        \underbrace{
        \begin{pmatrix}
            \sigma_1 &        &          \\
                     & \ddots &          \\ 
                     &        & \sigma_l
        \end{pmatrix}
        }_{\bm{\Sigma}^{(1)}}
        \underbrace{
        \begin{pmatrix}
            \horzbar & \bm{v}_1 & \horzbar \\
                      & \vdots   &  \\
            \horzbar & \bm{v}_l &  \horzbar \\
        \end{pmatrix} 
        }_{\bm{V}^{(1)^T}}
    \end{align}
    This is the reduced singular value decomposition of matrix. 
    However, in many cases we would like the outer matrices to be 
    square and orthogonal. 
    We can achieve this via padding zeros in the diagonal matrix and 
    adding completing the rows/columns of the outer matrices to form full bases.
    
    We first look at the term 
    \begin{align}
        \bm{DV}^{(1)^T} = 
        \begin{pmatrix}
        \vertbar &        & \vertbar \\
        \bm{\sigma}_1 & \hdots & \bm{\sigma}_l \\
        \vertbar &        & \vertbar 
        \end{pmatrix}
        \begin{pmatrix}
            \horzbar & \bm{v}_1 & \horzbar \\
                      & \vdots   &  \\
            \horzbar & \bm{v}_l &  \horzbar \\
        \end{pmatrix}  = \sum_{i=1}^l \bm{\sigma_i} \bm{v_i}^T
    \end{align}
    Similar to above we can add rows to $\bm{V}^{(1)^T}$ so that it's rows form a
    complete set of orthonormal vectors. If we add zero columns to $\Sigma$ the overall product
    will not change as is evident from the outer product view of matrix multiplication.
    
    \begin{align}
        \bm{DV}^{(1)^T} =& \sum_{i=1}^l \bm{\sigma_i} \bm{v_i}^T                                      \\
                        =& \sum_{i=1}^l \bm{\sigma_i} \bm{v_i}^T + \sum_{i=l+1}^n \bm{0} \bm{v_i}^T   \\
                        =& \begin{pmatrix}
                                \vertbar      &        & \vertbar      & \vertbar          &        & \vertbar      \\
                                \bm{\sigma}_1 & \hdots & \bm{\sigma}_l & \bm{0}_{l+1} & \hdots & \bm{0}_n \\
                                \vertbar      &        & \vertbar      & \vertbar          &        & \vertbar 
                            \end{pmatrix}
                            \begin{pmatrix}
                                \horzbar & \bm{v}_1 & \horzbar \\
                                        & \vdots   &           \\
                                \horzbar & \bm{v}_l & \horzbar \\
                                \horzbar & \bm{v}_{l+1} & \horzbar \\
                                         & \vdots   &          \\
                                \horzbar & \bm{v}_n & \horzbar 
                            \end{pmatrix} \\
                        =& \bm{\Sigma}^{(2)} \bm{V}^T
    \end{align}
    We have added some columns to $\bm{\Sigma}^{(2)}$ which has a shape of $(l \x n)$. 
    $\bm{V}$ is of shape $(n \x n)$ and now contains a complete set of orthonormal vectors.
    Similar to the above we can add columns to $\bm{U}^{(1)}$ and compensate them via adding
    zero-rows to $\Sigma^{(2)}$.
    \begin{align}
        \bm{U}^{(1)} \bm{\Sigma}^{(2)} =&
        \begin{pmatrix}
        \vertbar &        & \vertbar \\
        \bm{u}_1 & \hdots & \bm{u}_l \\
        \vertbar &        & \vertbar 
        \end{pmatrix}
        \begin{pmatrix}
            \horzbar & \bm{\sigma}_1     & \horzbar \\
                     & \vdots            &          \\
            \horzbar & \bm{\sigma}_l     & \horzbar \\
        \end{pmatrix} \\
        =&\begin{pmatrix}
            \vertbar &        & \vertbar & \vertbar &        & \vertbar \\
            \bm{u}_1 & \hdots & \bm{u}_l & \bm{u}_{l+1} & \hdots & \bm{u}_m \\
            \vertbar &        & \vertbar & \vertbar &        & \vertbar \\
            \end{pmatrix}
            \begin{pmatrix}
                \horzbar & \bm{\sigma}_1     & \horzbar  \\
                         & \vdots            &           \\
                \horzbar & \bm{\sigma}_l     &  \horzbar \\
                \horzbar & \bm{0}_{l+1} & \horzbar  \\
                         & \vdots            &           \\
                \horzbar & \bm{0}_m     &  \horzbar \\
        \end{pmatrix} \\
        =& \bm{U \Sigma}
    \end{align}
    \allowdisplaybreaks
    
    This yields the SVD in its final form:
    \begin{align}
        \bm{M} = \bm{U \Sigma V^T} =
         \begin{pmatrix}
            \vertbar &        & \vertbar \\
            \bm{u}_1 & \hdots & \bm{u}_m \\
            \vertbar &        & \vertbar 
        \end{pmatrix}
        \overset{\quad  l \x l \quad \quad \quad\quad \blue{l \x (n-l)}}{
        \underset{\ \ \ \red{(m-l) \x l} \ \ \quad \green{(m-l) \x (n-l)}}{
        \begin{pmatrix}
                \sigma_1 &        &          & \blue{0      } & \blue{\hdots}  &  \blue{0}  \\
                         & \ddots &          & \blue{\vdots } & \blue{\ddots }  & \blue{\vdots }  \\
                         &        & \sigma_l & \blue{0      } & \blue{\hdots }  & \blue{0      }  \\
                \red{0     }   & \red{\hdots} & \red{0     } & \green{0     }  & \green{\hdots} & \green{0} \\
                \red{\vdots}   & \red{\ddots} & \red{\vdots} & \green{\vdots}  & \green{\ddots} & \green{\vdots}   \\
                \red{0     }   & \red{\hdots} & \red{0     } & \green{0     }  & \green{\hdots} & \green{0}
        \end{pmatrix}}}
        \begin{pmatrix}
            \horzbar & \bm{v}_1     & \horzbar \\
                     & \vdots            &          \\
            \horzbar & \bm{v}_n     & \horzbar \\
        \end{pmatrix},
    \end{align}
    The blue entries were added when we augmented the $\bm{v}_i$ and the red and green zeros when we 
    completed the $\bm{u}_i$.


\section{Miscellaneous}
\subsection{A real matrix with complex eigenvalues}
A real matrix can have complex eigenvalues
\begin{equation*}
\begin{array}{l}
  \begin{pmatrix}
      0 & 1 \\
      -1 & 0
  \end{pmatrix}
  \Vec{v} = \lambda\Vec{v}\\
  \\
  \begin{pmatrix}
      -\lambda & 1 \\
      -1 & -\lambda
  \end{pmatrix}
  \Vec{v} =0\\
  \\
  \det
  \begin{pmatrix}
      -\lambda & 1 \\
      -1 & -\lambda
  \end{pmatrix}
  =0\\
  \lambda^2+1=0
  \hspace{0.5cm}
  \lambda = \pm i
  \hspace{0.5cm}
  v_{1}=
  \begin{pmatrix}
      1 \\
      i
  \end{pmatrix}\\
  \vspace{0.5cm}
  \begin{pmatrix}
      -i & 1 \\
      -i & 1 
  \end{pmatrix}
  \begin{pmatrix}
      x_{1} \\
      y_{2}
  \end{pmatrix}
  =
  \begin{pmatrix}
      0 \\
      0 
  \end{pmatrix}\\
  -ix_{1}+y_{2} =0\\
  x_{1} =1\\
  y_{2} =i
\end{array}
\end{equation*}


Note:
\begin{equation*}
  \begin{array}{l}
    \begin{pmatrix}
        i & 1 \\
        -1 & i
    \end{pmatrix}
    \begin{pmatrix}
        x \\
        y
    \end{pmatrix}
    =
    \begin{pmatrix}
        0 \\
        0
    \end{pmatrix}\\
    x_{i}+y =0 \\
    x=1\\
    y=-i\\
    \Vec{v_{2}} =
    \begin{pmatrix}
      1 \\
      -i
    \end{pmatrix}\\
  \end{array}
\end{equation*}

Complex number: z=a+ib
Complex conjugate is defined as $\bar{z}=a-ib$
$z\bar{z} = (a+ib)(a-ib) = a^2 + b^2 \geq 0$

\begin{equation*}
  \begin{array}{l}
  \text{Complex vector  } \Vec{v}=
    \begin{pmatrix}
        z_{1} \\
        z_{2} \\
        z_{3} \\
        \vdots \\
        z_{n}
    \end{pmatrix}
  \end{array}
\end{equation*}

\begin{center}
    $\bar{\Vec{v}} \cdot \Vec{v} \geq 0$ with equality iff $\Vec{v} = \Vec{0}$
\end{center}
\vspace{20px}
\chapter{Summarizing univariate data}
There are n observations, where n is the number of students.
It is hard to make sense of this data. 
First idea is analyzing the data: 
Calculate size of a typical person by so called "central tendency".

\begin{center}
    $ X=(160, 163, 156, \ldots, 183) = (x_{1},\ldots, x_{n})$
\end{center}
    
\section {Arithmetic mean}

\begin{align}
\bar{x} = \frac{1}{n}\sum_{i=1}^{n}x_{i} = \frac{1}{n}\left(x_{1}+x_{2}+\ldots+x_{n}\right)
\end{align}

Not a robust statistic because it is heavily influenced by outliers.
\paragraph{Example:} 100 people live in a village. Everybody is earning 100 \texteuro \hspace{0.1cm} a month, but one person earns 1 billion \texteuro \hspace{0.1cm} a month. The average wage is 10099 \texteuro, but the typical person is not earning that much. 
    
    
\subsection {Median}
More robust measure of central tendency.
The median splits the data sample in half. 

\begin{align}
      median(x) = 
      \begin{cases}
          x_{\frac{n+1}{2}} & \text{if n is odd}\\
          \frac{1}{2}(x_{n/2}+x_{n/2+1}) & \text{if n is even}
      \end{cases}
\end{align}       
    
    
\paragraph{Example:}

\begin{align}
    x &= (100,110,120,200,300,1000000)\\
    median(x) &= \frac{120+200}{2} = 160
\end{align}
    
\subsection {Quantiles}
Cut points that divide sample into equally sized slices. 
Quantiles divide the data into four regions.
\begin{itemize}
  \item $Q_{1}$: the first quantile, middle number between minimum and median
  \item $Q_{2}$: the second quantile, median  
  \item $Q_{3}$: the third quantile, middle number between median and maximum
\end{itemize}


Similarly deciles split the data into ten and percentiles into 100 slices.
    
\subsection {Boxplot}
Popular use of quantiles. It gives information both about the central tendency and spread of the data.

% use figure env
% \begin{figure}
% \includegraphics[scale=0.2]{boxplot.jpg}
% \end{figure}
    
Whisker options:
\begin{itemize}
\item minimum / maximum
\item $l=1.5 \ (Q_{3}-Q_{1})$, values outside of the whiskers are called outliers and drawn individually
\item e.g. can show the $5^{th}$ and $95^{th}$ percentile
\end{itemize}
    
    
\subsection {Histogram}
More complex than boxplot but better suited for multi-modal distributions, i.e. distributions with several peaks:
\begin{itemize}
  \item Divide data range into equally sized intervals
  \item Count number of samples in each interval
\end{itemize}
    
% \begin{figure}
% \includegraphics[scale=0.6]{mulitmodal1.png}
% \end{figure}
The interval size has to be chosen suitably.

\begin{itemize}
    \item Too small $\Rightarrow$ all counts are either one or zero. 
    \item Too large $\Rightarrow$ no gained information.
\end{itemize}
    
\subsection {Measuring spread of data}
A common way to measure the spread of data is to calculate the average distance to the mean value.
A simple idea idea would be to use:

\begin{align}
   & \frac{1}{n}\sum_{i=1}^{n}(x_{i}-\bar{x}) \\
  =& \frac{1}{n}\left(\sum_{i=1}^{n}x_{i} - \sum_{i=1}^{n}\bar{x}\right)\\
  =& \frac{1}{n}\left(\sum_{i=1}^{n}x_{i} - n\bar{x}\right) \\
  =& \frac{1}{n}\left(\sum_{i=1}^{n}x_{i}\right) - \bar{x} \\
  =& \bar{x}-\bar{x} = 0
\end{align}
    
\subsection {Mean absolute deviation}
\begin{align}
  \text{MAD}(x) = \frac{1}{n}\sum_{i=1}^{n}|(x_{i}-\bar{x})|
\end{align}
    
Exercise: Find a that minimizes
\begin{align}
    \frac{1}{n}\sum_{i=1}^{n}|(x_{i}-a)|
\end{align}
    
    
\subsection {Variance: Mean squared deviation}

\begin{align}
\sigma^2(x) = \text{var}(x) = \frac{1}{n}\sum_{i=1}^{n}\left(x_{i}-\bar{x}\right)^2
\end{align}

Alternative way to write this:
\begin{align}
\text{var}(x) =& \frac{1}{n}\left(\sum_{i=1}^{n}(x_{i}^2 - 2x_{i}\bar{x}+\bar{x}^2)  \right) =\\
         =& \frac{1}{n}\left(\sum_{i=1}^{n} x_{i}^2 - 2\bar{x} \ \frac{1}{n}\sum_{i=1}^{n}x_{i}+\bar{x}^2 \right) =\\
         =& \frac{1}{n}\left(\sum_{i=1}^{n} x_{i}^2 - 2\bar{x}^2+\bar{x}^2 \right) =\\
         =& \frac{1}{n}\left(\sum_{i=1}^{n} x_{i}^2 - \bar{x}^2 \right) = \\ 
         =& \left(\frac{1}{n} \sum_{i=1}^{n} x_{i}^2 \right) - \bar{x}^2  = \\ 
         =& \bar{x^2} - \bar{x}^2
\end{align}
This means the variance can be calculated using the mean squared value and the mean of the data.

\paragraph{Exercise:} Find the value of $a$ that minimizes 
\begin{align}
\frac{1}{n}\sum_{i=1}^{n}(x_{i}-a)^2
\end{align}


\subsection {Standard deviation}

\begin{equation*}
\sigma(x) = \text{std}(x) = \sqrt{\text{var}(x)} =\sqrt{\frac{1}{n}\sum_{i=1}^{n}(x_{i}-\bar{x})^2}
\end{equation*}


\section{Summarizing Bivariate Data}
When dealing with bivariate data we have two values for each sample.
For example, we measure weight ($x_i$) and height ($y_i$) of animal with index $i$.
Our dataset is then given by tuples of data:
\begin{align}
  z = (z_{1},z_{2},\ldots,z_{n}) =((x_{1},y_{1}),(x_{2},y_{2}),(x_{3},y_{3}),\ldots,(x_{n},y_{n}))
\end{align}

\subsection{Scatterplot}
    
\subsection {Covariance}
To analyze this kind of data we can again look at the central tendencies and 
spreads of the single features.
Additionally we can analyze how the two features they vary together.

The covariance between $x$ and $y$ is defined as

\begin{align}
s_{xy}=\frac{1}{n}\sum_{i=1}^{n}(x_{i}-\bar{x})(y_{i}-\bar{y}) = s_{yx}
\end{align}
If the covariance is positive then on average then on avarage a 
positive deviation from the mean in one feature leads to one in the other and vice versa.
A negative value says that a positive deviation from the mean in one feature leads to a negative one in the other.

Similar to the variance we can also write the covariance as 
    
\begin{align}
    s_{xy} &= \frac{1}{n} \sum_{i=1}^{n} (x_{i}y_{i}-x_{i}\bar{y}-\bar{x}y_{i}+\bar{x}\bar{y}) \\
           &= \frac{1}{n} \sum_{i=1}^{n} x_{i}y_{i}-\bar{y}\frac{1}{n}\sum_{i=1}^{n}x_{i}-\bar{x}\frac{1}{n}\sum_{i=1}^{n}y_{i}+\bar{x}\bar{y} \\
           &= \frac{1}{n} \sum_{i=1}^{n} x_{i}y_{i}-\bar{x}\bar{y}-\bar{x}\bar{y}+\bar{x}\bar{y}\\
           &= \frac{1}{n} \sum_{i=1}^{n} x_{i}y_{i}-\bar{x}\bar{y}
\end{align}



The covariance of a feature with itself recovers the variance.
The variance on covariance are often combined in the so called covariance matrix
\begin{equation*}
\bm{\Sigma} = 
\begin{pmatrix}
s_{xx} & s_{xy} \\
s_{yx} & s_{yy} 
\end{pmatrix}
\end{equation*}
It gives information about spread of data ($s_{xx},s_{yy}$ and how they vary together.

\newcommand*\mean[1]{\overline{#1}}

The data is often written in form of a matrix 
\begin{align}
\bm{Z} = 
\begin{pmatrix}
x_{1} & x_{2} & \hdots & x_n \\
y_{1} & y_{2} & \hdots & y_{n} \\
\end{pmatrix} 
\end{align}
and by taking the row-wise means we get the vector of means
\begin{equation*}
\mean{\bm{z}} = 
\begin{pmatrix}
\bar{x} \\
\bar{y} 
\end{pmatrix}
\end{equation*}

We can take another look at the covariance matrix and try to write it in the terms defined above.
\begin{align}
  \bm{\Sigma} =& \frac{1}{n} \begin{pmatrix}
    \sum_{i=1}^{n}x_{i}^2 - \mean{x}^2      & \sum_{i=1}^{n}x_{i}y_{i} - \mean{x} \ \mean{y} \\
    \sum_{i=1}^{n}y_{i}x_{i} - \mean{x} \ \mean{y} & \sum_{i=1}^{n}y_{i}y_{i} -  \mean{y}^2
  \end{pmatrix} = \\
=& \frac{1}{n} \red{\begin{pmatrix}
  \sum_{i=1}^{n}x_{i}^2    & \sum_{i=1}^{n}x_{i}y_{i} \\
  \sum_{i=1}^{n}y_{i}x_{i} & \sum_{i=1}^{n}y_{i}y_{i} 
\end{pmatrix}} - 
\blue{\begin{pmatrix}
  \mean{x}^2        &  \mean{x} \mean{y} \\
  \mean{x} \mean{y} &  \mean{y}^2
\end{pmatrix}} = \\
=& \frac{1}{n}\red{
\begin{pmatrix}
    x_{1} & x_{2} & \hdots & x_{n} \\
    y_{1} & y_{2} & \hdots & y_{n}
\end{pmatrix}
\begin{pmatrix}
    x_{1} & y_{1} \\
    x_{2} & y_{2} \\
    \vdots  & \vdots \\
    x_{n} & y_{n}
\end{pmatrix}} - \blue{\begin{pmatrix}
  \mean{x} \\
  \mean{y}
\end{pmatrix} \begin{pmatrix}
  \mean{x} &
  \mean{y}
\end{pmatrix}} = \\
=& \frac{1}{n}\red{\bm{ZZ}^T}- \blue{\mean{\bm{z}} \ \mean{\bm{z}}^T}
\end{align}

\begin{equation*}
S_{xy}=\frac{1}{n}\sum_{i=1}^{n}x_{i}y_{i}-\bar{x}\bar{y}
\end{equation*}
    
    
\subsection {Pearson Correlation Coefficient}
The pearson correlation coefficient 
\begin{align}
    r_{xy} = \frac{\sum_{i=1}^{n}(x_{i}-\bar{x})(y_{i}-\bar{y})}{\sqrt{\sum_{i=1}^{n}(x_{i}-\bar{x})^2}\sqrt{\sum_{i=1}^{n}(y_{i}-\bar{y})^2}}
\end{align}
can be seen as a normalized covariance.
It ranges from $-1$ to $1$.
  To see this, we view the quantities above as vectors.
\begin{align}
  u_{i} = (x_{i}-\mean{x}) \\
  v_{i} = (y_{i}-\mean{y})
\end{align}
  
  
\begin{align}
      \bm{u} = 
        \begin{pmatrix}
      x_{i} \\
      \vdots \\
      x_{n}
        \end{pmatrix}
        -\bar{x}
        \hspace{40pt}
      \bm{v} =  
        \begin{pmatrix}
      y_{i} \\
      \vdots \\
      y_{n}
        \end{pmatrix}
        -\bar{y}\\
\end{align}
  
\begin{equation*}
r_{xy} = \frac{\bm{u} \cdot \bm{v}}{\norm{\bm{u}}\norm{\bm{v}}} = cos_{\alpha}
\end{equation*}

    
\begin{align}
  r_{xy} =
    \begin{cases}
      1  & \text{if $x_{i}= ay_{i}+b$ with $a>0$} \\
      -1 & \text{if $x_{i}= -ay_{i}+b$ with $a>0$}
    \end{cases}       
\end{align}
    
For proof see Cauchy-Schwarz inequality.
    
\subsection {Linear regression}
In linear regression our goal is to predict the value 
of one variable given that of another using a linear relation.
Given the data $\left\{(x_{1},y_{1}),\ldots, (x_{n},y_{n})\right\}$
we want to find a function that predicts the value of $y_i$ given that of $x_i$.
This prediction should be made by the model
\begin{equation*}
  \hat{y}_i = wx_i+b
\end{equation*}
where $w$ and $b$ are parameters that have to be adjusted to fit the data.
A simple example would be to predict the weight of a dog given its height.
If all the data points would lie on a straight line we could easily calculate the values
of $w$ and $b$ that fit the data.
However, if this is not possible we can try to find parameters that 
minimize the mean squared error:

\begin{align}
  L(w,b) =& \frac{1}{n}\sum_{i=1}^{n}(y_{i}-\hat{y_{i}})^2 \\
         =& \frac{1}{n}\sum_{i=1}^{n}(y_{i}-(wx_{i}+b))^2
\end{align}

To find a minimum we first set the gradient of the loss w.r.t. to the parameters to zero.

\begin{equation*}
  \bm{\nabla} L(w,b) =  
  \begin{pmatrix}
      \frac{\partial L(w,b)}{\partial w} \\
      \frac{\partial L(w,b)}{\partial b} 
  \end{pmatrix}
  =
  \begin{pmatrix}
      0 \\
      0 
  \end{pmatrix}
\end{equation*}

This gives us a system of linear equations:
\begin{align}
  \frac{\partial L}{\partial w} =& \frac{1}{n}\sum_{i=1}^{n}-2(y_{i}-(wx_{i}+b))x_{i}\\
  \frac{\partial L}{\partial b} =& \frac{1}{n}\sum_{i=1}^{n}-2(y_{i}-(wx_{i}+b))
\end{align}

Solving for $w$ and $b$:
\begin{align}
  &\frac{1}{n}\left(\sum_{i=1}^{n}-2(y_{i}-(wx_{i}+b))\right)=0\\
  &\frac{1}{n}\sum_{i=1}^{n}y_{i}-w\frac{1}{n}\sum_{i=1}^{n}x_{i}-b=0\\
  &\bar{y}-w\bar{x} =b
\end{align}

With the result for $b$ we can solve for $w$:
\begin{align}
    & \frac{1}{n}\sum_{i=1}^{n}-2(y_{i}-(wx_{i}+b))=0 \\
    & \frac{1}{n}\sum_{i=1}^{n}-2(y_{i}-(wx_{i}+b))x_{i}=0\\
    & \frac{1}{n}\sum_{i=1}^{n}y_{i}x_{i}-\bar{x}\bar{y}+\bar{x}\bar{y}-w(\frac{1}{n}\sum_{i=1}^{n}x_{i}^2-\bar{x}^2+\bar{x}^2)-b\frac{1}{n}\sum_{i=1}^{n}x_{i} \\
    & = s_{xy} + \bar{x}\bar{y}-w s_{x}-w\bar{x}^2-b\bar{x}\\
    & \mid b=\bar{y}-w\bar{x}\mid\\
    & = s_{xy}+\bar{x}\bar{y}-w s_{x}-w\bar{x}^2-\bar{x}\bar{y}+w\bar{x}^2\\
    & = s_{xy} -w s_{x}\\
    & =0
\end{align}

Thus we get 
\begin{align}
  s_{x} w = s_{xy}
\end{align}
and
\begin{align}
w =& \begin{cases}
    \frac{s_{xy}}{s_{x}} & \text{if $s_{x} \neq 0$ }\\
    \text{arbitrary} & \text{if $s_{x}$ = 0 }
  \end{cases}\\
  b=&\bar{y}-w\bar{x} 
\end{align}

Next we want to check if this corresponds to a minimum. We do so 
by checking the Hessian matrix is positive-definite.
The first derivatives are:
\begin{align}
    \frac{\partial L}{\partial w} &= \frac{1}{n}\sum_{i=1}^{n}-2(y_{i}-(wx_{i}+b))x_{i}\\
    \frac{\partial L}{\partial b} &= \frac{1}{n}\sum_{i=1}^{n}-2(y_{i}-(wx_{i}+b))\\
\end{align}

Further differentiation gives:
\begin{align}
    \pdv[2]{L}{w}                 &= -\frac{2}{n}\sum_{i=1}^{n}-x_{i}^2 = 2 \mean{x^2} \\
    \pdv{L}{w}{b}                 &= \frac{2}{n}\sum_{i=1}^{n}x_{i} = 2\bar{x}\\
    \pdv[2]{L}{b}                 &= 2
\end{align}

In general it holds that 
\begin{align}
  \frac{\partial}{\partial b} \left(\frac{\partial L}{\partial w}\right)= \frac{\partial}{\partial w} \left(\frac{\partial L}{\partial b}\right),
\end{align}
which lets us save some computation.

This gives us:
\begin{equation*}
  H(L) = 2 
  \begin{pmatrix}
      \bar{x^2} & \bar{x} \\
      \bar{x} & 1 
  \end{pmatrix}
\end{equation*}
If this matrix is strictly positive definite
we have indeed obtained a local minimum.

\begin{align}
  &\begin{pmatrix}
    v_{1}  & v_{2}
  \end{pmatrix}
  \begin{pmatrix}
    \mean{x^2}  & \mean{x} \\
    \mean{x}  & 1
  \end{pmatrix}
  \begin{pmatrix}
    v_{1} \\
    v_{2}
  \end{pmatrix}=\\
  &v_{1}^2\mean{x^2} + 2v_{1}v_{2} \mean{x}+v_{2}^2 \geq \\
  & v_{1}^2\mean{x}^2 + 2v_{1}v_{2}\bar{x}+v_{2}^2 = \\
  &(v_{1}\bar{x}+v_{2})^2\geq0
\end{align}
where we have used 
\begin{equation}
  \mean{x^2}-\mean{x}^2 = \text{var}(x)\geq 0 \Rightarrow
  \mean{x^2} \geq \mean{x}^2
\end{equation}
if $\text{var}(x)>0$ this becomes a strict inequality which means that we get a unique minimum.
Recall that when we have no variance in the $x_i$ the value for $w$ was arbitrary.

Thus we have established a solution for the linear regression problem.

\chapter{Principal component analysis}
\paragraph{Motivating example}
There is a park and we are given the locations of the trees in it
given two numbers.
It appears that they are approximately planted in a line with slight deviations.
However we would like to find a single direction which tells us the most about the location of tree.
In the figure it is evident that if we can get a very good estimate of where trees are,
by knowing how far to go along the red line. The (blue) distance orthogonal to it is of less importance.
If we encoded each point by it's red and blue distance we would find 
that the red ones vary a lot more than the blue ones.
Thus we will define the most important direction as the one along which the data has the highest variance.

\section{Variance maximization}
Given a vector $\bm{v}$ with unit length  $||\bm{v}||=1$ we can calculate the 
coordinate wrt. to it via a dot product,
\begin{equation}
    y_i = \bm{x}_i \cdot \bm{v}, 
\end{equation}
yielding a scalar for each data-point.
The variance of the values $\{y_1, \hdots, y_n\}$ is 
\begin{align}
    &\frac{1}{n} \sum_{i=1}^n y_i^2 - \bar{y}^2 \\
    =& \frac{1}{n} \sum_{i=1}^n (\bm{x}_i \bm{v})^2 - \left(\frac{1}{n} \sum_{i=1}^n \bm{x}_i\bm{v}\right)^2 =& \text{def. of $y_i$} \\
    =&\frac{1}{n} \sum_{i=1}^n \sum_{j=1}^m \sum_{k=1}^m x_{ij} v_j x_{ik} v_k - \bm{v}^T \bm{\mu} \bm{\mu}^T \bm{v} =&\text{write dot product as sum}\\
    =&\frac{1}{n}  \sum_{j=1}^m \sum_{k=1}^m v_j  \red{\sum_{i=1}^n x^T_{ji}  x_{ik}} v_k - \bm{v}^T \bm{\mu} \bm{\mu}^T \bm{v} =&\text{rearrange}\\
    =&\frac{1}{n}  \sum_{j=1}^m \sum_{k=1}^m v_j  \red{[\bm{X^T X}]_{ik}} v_k - \bm{v}^T \bm{\mu} \bm{\mu}^T \bm{v} =&\text{replace sum by m.m.}\\
    =&\frac{1}{n}  \sum_{j=1}^m \sum_{k=1}^m v_j  [\bm{X^T X}]_{ik} v_k - \bm{v}^T \bm{\mu} \bm{\mu}^T \bm{v} =&\text{replace sum by m.m.}\\
    =&\frac{1}{n}  \bm{v}^T \bm{X^T X} \bm{v}- \bm{v}^T \bm{\mu} \bm{\mu}^T \bm{v} =&\text{replace sums by m.m.}\\
    =&\bm{v}^T \left(\frac{1}{n}\bm{X^T X} - \bm{\mu} \bm{\mu}^T \right) \bm{v} =&\text{rearrange}\\
    =&\bm{v}^T \bm{C} \bm{v} & \text{def. of covariance}
\end{align}
Since $\bm{C}$ is real and symmetric we can express it using it's eigenvalues and eigenvectors,
\begin{equation}
    \bm{C} = \sum_{j=1}^m \lambda_j \bm{w}_j \bm{w}_j^T.
\end{equation}
Since $(w_1, \hdots, w_m)$ form an orthonormal basis we can also write $\bm{v}$ as linear combination of them,
\begin{equation}
    \bm{v} = \sum_{i=1}^m \alpha_i \bm{w}_i.
\end{equation}
Thus we shift problem of finding the entries of $\bm{v}$ to finding the $\alpha_i$ that maximize the variance.
We can write our objective as 
\begin{align}
    \bm{v^T} \bm{C v} &= \sum_{i=1}^m \alpha_i \bm{w}_i^T \sum_{j=1}^m \bm{w}_j\bm{w}_j^T \lambda_j \sum_{k=1}^m \alpha_k \bm{w}_k^T &\\
    &= \sum_{i=1}^m \sum_{j=1}^m \sum_{k=1}^m \alpha_{i} \alpha_{k} \lambda_{j} \red{\bm{w}_i^T \bm{w}_j} \blue{\bm{w}_j^T \bm{w}_k} &\\
    &= \sum_{i=1}^m \sum_{j=1}^m \sum_{k=1}^m \alpha_{i} \alpha_{k} \lambda_{j} \red{\delta_{ij}} \blue{\delta_{jk}}&\\
    &= \sum_{i=1}^m \sum_{j=1}^m  \alpha_{i} \alpha_{j} \lambda_{j} \delta_{ij} & \text{sum over $k$} \\ 
    &= \sum_{i=1}^m \alpha_{i} \alpha_{i} \lambda_{i}  & \text{sum over $j$} \\ 
    &= \sum_{i=1}^m \alpha_{i}^2 \lambda_{i}.
\end{align}

Since $||\bm{v}|| = 1$ we know that $\sum_{i=1}^m \alpha_i^2 = 1$. 
The above result gives us a weighted average which is maximal when $\alpha_1=1$ since $\lambda_1$ is (one of) the largest eigenvalue(s).
Thus the direction in which the variance of our data is maximal is just the direction of the eigenvector with the largest eigenvalue.

We now want to find the second most important direction 

\red{Problem: Why do we want the directions to be orthogonal.}









% \chapter{Linear algebra: Singular value decomposition}
% \subsection{Another look at PCA}
% \section{Multiple linear regression}
% \section{Probability theory}
% \section{}



\end{document}