\chapter{Linear algebra: Basics}

In data analysis we will often represent our data as matrices. Consequently many algorithms are described using
the concept of matrices and their properties which could be broadly called the study of linear algebra. Here I
don’t want to give an axiomatic introduction but rather provide some basic results with the prerequisite that
the reader already knows some basics about vectors and matrices.
We will denote vectors as bold lowercase letters 
\begin{equation}
\bm{a} = 
\begin{pmatrix}
a_1     \\
\vdots  \\
a_n
\end{pmatrix},
\end{equation}
where $a_i$ is the $\text{i}^{th}$ entry of the vector $\bm{a}$.

Matrices are denoted by bold uppercase letters
\begin{equation}
    \bm{A} = 
    \begin{pmatrix}
        a_{11} & \hdots & a_{1n} \\ 
        \vdots & \ddots & \vdots \\
        a_{m1} & \hdots & a_{mn}
    \end{pmatrix}.
\end{equation}
Depending on the context it is useful to view a matrix as a concetation of either row or column vectors.
\begin{align}
    \bm{A} = \begin{pmatrix}
        \horzbar & \bm{a}_1^r & \horzbar \\
                  & \vdots   &  \\
        \horzbar & \bm{a}_m^r &  \horzbar \\
    \end{pmatrix} = 
    \begin{pmatrix}
    \vertbar &        & \vertbar \\
    \bm{a}_1^c & \hdots & \bm{a}_n^c \\
    \vertbar &        & \vertbar 
    \end{pmatrix}.
\end{align}
We made an explicit distinction between row- and column vectors using superscripts here but these will often be left out
for better readability.

\section{Matrix multiplication}
Given two matrices 
\begin{align}
    \bm{A} = 
    \begin{pmatrix}
        a_{11} & \hdots & a_{1n} \\ 
        \vdots & \ddots & \vdots \\
        a_{m1} & \hdots & a_{mn}
    \end{pmatrix} \in R^{m \times n}, &&
    \bm{B} = 
    \begin{pmatrix}
        b_{11} & \hdots & b_{1p} \\ 
        \vdots & \ddots & \vdots \\
        b_{n1} & \hdots & b_{np}
    \end{pmatrix} \in R^{n \times p}
\end{align}
their product $\bm{C} = \bm{A}\bm{B}$ is of shape $(m \times p)$ and it's entries are
\begin{equation}
    c_{ik} = \sum_{j=1}^n a_{ij} b_{jk}
\end{equation}

\newcommand{\x}{\times}
The resulting matrix $\bm{C}$ has a shape of $(m \x p) \leftarrow (m \x n)(n \x p)$. The inner dimension $n$ is "eliminated" by summing over it. One can also see that the sum moves horizontally/rowwise over the entries of A and down the
columns of B

Often we are interested in a special case in which one of the matrices reduces to a vector, that means it has only
one column or one row. We define a column vector as a shorthand notation for a matrix $\bm{B} \in R^{n \x 1}$ with only one column and drop the column index:

\begin{equation}
    \bm{B} =
    \begin{pmatrix}
    b_{11} \\
    \vdots \\
    b_{n1}
    \end{pmatrix}     \equalhat
    \begin{pmatrix}
        b_{1} \\
        \vdots \\
        b_{n}
    \end{pmatrix} = \bm{b}
\end{equation}

Implicitly when we talk of a vector b with n elements we mean a matrix $\bm{B} \in R^{n\x 1}$. When we want to talk
about a matrix with only one row we write this a transposed vector
\begin{equation}
    \bm{b^T} = (b_1 \hdots b_n) \equalhat (b_{11} \hdots b_{1n})
\end{equation}

\subsection{Vector-vector products}
Two vectors can be multiplied with each other in two different ways. 
The scalar or inner product is defined as 
\begin{equation}
    \bm{a} \cdot \bm{b} = \bm{a}^T \bm{b} = \sum_{i=1}^n a_i b_i 
\end{equation}
This can be seen as a multiplication of matrices with shapes $(1 \x m)(m \x 1) \rightarrow (1 \x 1)$ and results in a scalar value.

The outer product between two vectors is 
\begin{equation}
    \bm{C} = \bm{a}\bm{b}^T = \begin{pmatrix}
    a_{1} \\
    \vdots \\
    a_{m}
    \end{pmatrix}
    \begin{pmatrix}
        b_1 & \hdots & b_n
    \end{pmatrix} = 
    \begin{pmatrix}
        a_1 b_1 & a_1 b_2 & \hdots & a_1 b_n \\
        a_2 b_1 & a_2 b_2 & \hdots & a_2 b_n \\
        \vdots & \vdots   & \ddots & \vdots  \\
        a_m b_1 & a_m b_2 & \hdots & a_m b_n
    \end{pmatrix}
\end{equation}

This means that every column is a multiple of $\bm{a}$ and every row a multiple of $\bm{b}$. 
We see that the entry $c_{ij} = a_i b_j$.

\subsection{Matrix-vector multiplication}
The product of a matrix and vector $\bm{c} = \bm{A}\bm{b}$ again is a vector. 
If $\bm{A} \in R^{m \x n}$ and $\bm{b} \in R^n$ their product
will be of shape $(m \x 1)$, or $\bm{c} \in R^m$.

\subsubsection{Linear combination of columns}

One can view matrix-vector product, $\bm{c} = \bm{Ab}$, as calculating a linear combination of the columns of $\bm{A}$ weighted by 
the entries of $\bm{b}$:


\begin{align}
\bm{Ab} &= \begin{pmatrix}
    \sum_{j=1}^n a_{1j}b_j\\
    \sum_{j=1}^n a_{2j}b_j\\
    \vdots \\
    \sum_{j=1}^n a_{mj}b_j\\
\end{pmatrix} = 
{\setlength\arraycolsep{1pt}
\begin{pmatrix}
    \red{a_{11}} b_1 &+& \blue{a_{12}} b_2 &+&\hdots &+& \green{a_{1n}} b_n \\
    \red{a_{21}} b_1 &+& \blue{a_{22}} b_2 &+&\hdots &+& \green{a_{2n}} b_n \\
    \red{\vdots} \ \ & & \blue{\vdots} \ \ & &\ddots & & \green{\vdots} \ \ \\
    \red{a_{m1}} b_1 &+& \blue{a_{m2}} b_2 &+&\hdots &+& \green{a_{mn}} b_n \\
\end{pmatrix}} = 
{\setlength\arraycolsep{1pt}
\begin{pmatrix}
\red{\vertbar} \ \ & & \blue{\vertbar} \ \ & &        & & \green{\vertbar} \ \ \\
\red{\bm{a}_1} b_1 &+& \blue{\bm{a}_2} b_2 &+& \hdots &+& \green{\bm{a}_n} b_n \\
\red{\vertbar} \ \ & & \blue{\vertbar} \ \ & &        & & \green{\vertbar} \ \
\end{pmatrix}}  \\
&=\red{\bm{a}_1} b_1 +  \blue{\bm{a}_2} b_2 + \hdots + \green{\bm{a}_n} b_n = \sum_{i=1}^n \bm{a}_i b_i
\end{align}

One prominent example where this view is useful is that if the columns of $\bm{A}$ form a basis.
Then a vector $\bm{c}$ can be written as $\bm{c} = \bm{A}\bm{c}_A$, where 
$\bm{c}_A$ are the coordinates of $\bm{c}$ with respect to the basis $\bm{A}$.
Equivalently if the columns of $\bm{B}$ form another basis then
\begin{equation}
    \bm{c} = \bm{A}\bm{c}_A = \bm{B}\bm{c}_B.
\end{equation}
From this equation one can easily get the prescription of calculating
the coordinates wrt. to $\bm{A}$ given those wrt. $\bm{B}$,

\begin{equation}
    \bm{c}_A = \bm{A}^{-1} \bm{B}\bm{c}_B.
\end{equation}

\subsubsection{Dot product}
Alternatively we can view a matrix vector product $\bm{Ab}$ as 
calculating the dot products of $\bm{b}$ with each row of $\bm{A}$,
\begin{equation}
    \bm{Ab} = \begin{pmatrix}
        \horzbar & \bm{a}_1 & \horzbar \\
                  & \vdots   &  \\
        \horzbar & \bm{a}_m &  \horzbar \\
    \end{pmatrix} 
    \begin{pmatrix}
    \vertbar \\
    \bm{b}  \\
    \vertbar
    \end{pmatrix}= 
    \begin{pmatrix}
    \bm{a}_{1} \cdot \bm{b} \\
    \vdots \\
    \bm{a}_{m} \cdot \bm{b} \\
    \end{pmatrix}
\end{equation}

\subsection{Matrix-matrix multiplication}
Now that we've looked at some special cases of matrix multiplication
we can go back to the general case 
$\bm{C} = \bm{AB}$.

Recalling the definition of matrix multiplication 
\begin{equation}
    c_{ik} = \sum_{j=1}^n a_{ij} b_{jk}
\end{equation}
we can look at the index $k$ and see that the $k$-th column of $\bm{C}$ is formed by the matrix vector product of $A$ and 
the $k$-th column vector of $\bm{B}$,
\begin{equation}
    \bm{C} = \bm{AB} = \bm{A} \begin{pmatrix}
    \vertbar &        & \vertbar \\
    \bm{b}_1 & \hdots & \bm{b}_p \\
    \vertbar &        & \vertbar 
    \end{pmatrix} = 
    \begin{pmatrix}
        \vertbar &        & \vertbar \\
        \bm{A}\bm{b}_1 & \hdots & \bm{A}\bm{b}_p \\
        \vertbar &        & \vertbar 
    \end{pmatrix} = 
\begin{pmatrix}
\vertbar &        & \vertbar \\
\bm{c}_1 & \hdots & \bm{c}_p \\
\vertbar &        & \vertbar 
\end{pmatrix}.
\end{equation}
Breaking matrix-matrix multiplication down to multiple matrix vector operations
allows us to view it as calculating multiple linear combinations of the columns of $\bm{A}$
or calculating the all inner products between the rows of $\bm{A}$ and the columns of $\bm{B}$,
\begin{equation}
    \bm{C} = \begin{pmatrix}
        \horzbar & \bm{a}_1 & \horzbar \\
                  & \vdots   &  \\
        \horzbar & \bm{a}_m &  \horzbar \\
    \end{pmatrix} 
    \begin{pmatrix}
    \vertbar &        & \vertbar \\
    \bm{b}_1 & \hdots & \bm{b}_p \\
    \vertbar &        & \vertbar 
    \end{pmatrix} = 
    \begin{pmatrix}
        \bm{a}_{1} \cdot \bm{b}_1 & \bm{a}_{1} \cdot \bm{b}_2 & \hdots & \bm{a}_{1} \cdot \bm{b}_p  \\
        \bm{a}_{2} \cdot \bm{b}_1 & \bm{a}_{2} \cdot \bm{b}_2 & \hdots & \bm{a}_{2} \cdot \bm{b}_p  \\
        \vdots                    & \vdots                    & \ddots & \vdots                     \\
        \bm{a}_{m} \cdot \bm{b}_1 & \bm{a}_{m} \cdot \bm{b}_2 & \hdots & \bm{a}_{m} \cdot \bm{b}_p  \\
    \end{pmatrix}.
\end{equation}
From this we can read that $c_{ik} = \bm{a}_i \bm{b}_j$.

\subsubsection{Sum of outer products}
Another view that is often useful is to 
write a matrix-matrix multiplication as a sum of outer products.
Looking again at our basic definition
\begin{equation}
    c_{ik} = \sum_{j=1}^n a_{ij} b_{jk} = a_{i1} b_{1k} + a_{i2} b_{2k} + a_{i3} b_{3k}
\end{equation}
we can see that for example the term $a_{i2} b_{2k}$ is the $ik$-th entry of the dot-product between
the second column vector of $\bm{A}$ and the second row vector of $\bm{B}$.
Thus we can deduce that the matrix $\bm{C}$ is a sum of outer-products,
\begin{equation}
    \bm{C} = \bm{AB} = \begin{pmatrix}
    \vertbar &        & \vertbar \\
    \bm{a}_1 & \hdots & \bm{a}_n \\
    \vertbar &        & \vertbar 
    \end{pmatrix}
    \begin{pmatrix}
        \horzbar & \bm{b}_1 & \horzbar \\
                  & \vdots   &  \\
        \horzbar & \bm{b}_n &  \horzbar \\
    \end{pmatrix}  = 
    \sum_{i=1}^n \bm{a}_i \bm{b}_i^T.
\end{equation}

\section{Orthogonal matrices}

An orthogonal matrix is a square matrix with the special property that its transposed is equal to its inverse.
That is for $\bm{U} \in \mathbb{R}^{m \x m}$
\begin{align}
    \bm{U^T} \bm{U} &= \bm{U} \bm{U^T} = \bm{1} \\
    \bm{U}^T &= \bm{U}^{-1}
\end{align}

Viewing this matrix-matrix product as pairwise dot-products between the columns/rows of the two matrices
reveals that the both the columns as well as the rows of $\bm{U}$ form an orthonormal set of vectors:
\begin{align}
    \begin{pmatrix}
        \horzbar & \bm{u}_1 & \horzbar \\
                  & \vdots   &  \\
        \horzbar & \bm{u}_m &  \horzbar \\
    \end{pmatrix} 
    \begin{pmatrix}
    \vertbar &        & \vertbar \\
    \bm{u}_1 & \hdots & \bm{u}_m \\
    \vertbar &        & \vertbar 
    \end{pmatrix} = 
    \begin{pmatrix}
        \bm{u}_{1} \cdot \bm{u}_{1} & \bm{u}_{1} \cdot \bm{u}_{2} & \hdots & \bm{u}_{1} \cdot \bm{u}_{m} \\
        \bm{u}_{2} \cdot \bm{u}_{1} & \bm{u}_{2} \cdot \bm{u}_{2} & \hdots & \bm{u}_{2} \cdot \bm{u}_{m} \\
        \vdots                    & \vdots                        & \ddots & \vdots                      \\
        \bm{u}_{m} \cdot \bm{u}_{1} & \bm{u}_{m} \cdot \bm{u}_{2} & \hdots & \bm{u}_{m} \cdot \bm{u}_{m} \\
    \end{pmatrix}   
    \begin{pmatrix}
        1 & 0 & \hdots & 0 \\
        0 & 1 & \hdots & 0 \\
        \vdots & \vdots & \ddots & \vdots \\
        0 & 0 & \hdots & 1
    \end{pmatrix}
\end{align}

The transformation can be seen as a rotation followed by reflection.
This is because angles and lengths are preserved under the transformation.
To see this take any two vectors $\bm{a}, \bm{b}$ and apply the transform $\bm{U}$.
We first establish that multiplication of a vector with $\bm{U}$ does not change its length:
\begin{align}
    \norm{\bm{Ua}}^2 = \bm{a}^T \red{\bm{U}^T\bm{U}} \bm{a} =\bm{a^T} \red{\bm{1}} \bm{a} = \norm{\bm{a}}^2
\end{align}

\begin{align}
    \cos{(\angle(\bm{Ua}, \bm{Ub}))} =&   \frac{\bm{U}\bm{a} \cdot \bm{U} \bm{b}}{\norm{\bm{Ua}} \ \norm{\bm{Ub}} }\\
    =& \frac{\bm{a}^T \bm{U^T} \bm{U} \bm{b}}{\norm{\bm{Ua}} \ \norm{\bm{Ub}} }\\
     =& \frac{\bm{a}^T \bm{b}}{\norm{\bm{a}} \norm{\bm{b}}} = \cos{(\angle(\bm{a}, \bm{b}))}.
\end{align}
\newcommand{\R}{\mathbb{R}}
\section{Eigenvalues and eigenvectors}
A square matrix $\bm{A} \in \R^{n \x n}$ can be seen as a linear map that maps a vector $\bm{v} \in \R^n$ to a vector $\bm{Av} \in \R^n$.
There are some vectors called \emph{eigenvectors} (derived from the german word "eigen") which are mapped onto a multiple of themselves,
\begin{align}
    \bm{Av} = \lambda \bm{v}.
    \label{eq:evdef}
\end{align}
If this equation holds $\bm{v}$ is called an eigenvector and $\lambda$ its associated eigenvalue.
The eigenvalues are a useful characteristic of a matrix and finding eigenvalues has lots of applications ranging from coupled oscillators, heat and Schrödinger equations in
physics to principal component analysis in data analysis.

We still need a way to find the eigenvalues/vectors. 
From \eqref{eq:evdef} we get 

\begin{align}
    \bm{Av} =& \ \lambda \bm{1}\bm{v} \\
    (\bm{A} - \lambda \bm{1})\bm{v} =& \ \bm{0}
\end{align}

This yields a linear system of equations, which only has a solution if the columns of 
$ \bm{A}-\lambda \bm{1}$ are linearly dependent.
Thus we can find eigenvalues by setting the determinant of this matrix to zero
\begin{align}
    \det(\bm{A} - \lambda \bm{1}) = 0    
\end{align}
This gives a polynomial of degree n, called the characteristic polynomial, which according to the fundamental
theorem of calculus has $n$ roots and can be written as
\begin{align}
    \det(\bm{A} - \lambda \bm{1}) = \prod_{i=1}^n (\lambda - \lambda_i),
\end{align}
where the $\lambda_i$ are solutions to the equation above. 
The number of times a specific value occurs in this factorization is called 
an eigenvalues algebraic multiplicity.
After calculating the eigenvalues one can insert them into \eqref{eq:??} one ofter another and solve for the eigenvectors
$v_i$. One interesting question is whether the set of all eigenvectors form a complete basis of $\R^n$. 
This is not the case if for an eigenvalue with algebraic multiplicity $n_a > 1$ the null space of $\bm{A} - \lambda_i$ has dimension less than $n_a$.


\paragraph{Example} Given a matrix
\begin{align}
\bm{A} = 
\begin{pmatrix}
    1 & 1 \\
    0 & 1
\end{pmatrix}
\end{align}
We can calculate it’s characteristic polynomial
\begin{align}
    \det(\bm{A} - \lambda \bm{1}) = (1-\lambda)^2
\end{align}
Thus we get an eigenvalue $\lambda = 1$ with algebraic multiplicity of two. However when inserting this eigenvalue to
solve for $\bm{v}$ we get
\begin{align}
    (\bm{A} - \lambda \bm{1})  \bm{v} =  \begin{pmatrix}
        0 & 1 \\
        0 & 0
    \end{pmatrix}
    \begin{pmatrix}
        v_1 \\
        v_2
    \end{pmatrix}
    \Rightarrow \bm{v} = s \begin{pmatrix}
        1 \\
        0
    \end{pmatrix}
\end{align}

\subsection{Diagonalization of a matrix}
If we find an eigenbasis, that means a basis that only consists of eigenvectors, we can write the following:
\begin{align}
    \bm{A} \begin{pmatrix}
    \vertbar &        & \vertbar \\
    \bm{v}_1 & \hdots & \bm{v}_n \\
    \vertbar &        & \vertbar 
    \end{pmatrix} = 
    \begin{pmatrix}
        \vertbar &        & \vertbar \\
        \lambda_1\bm{v}_1 & \hdots & \lambda_n\bm{v}_n \\
        \vertbar &        & \vertbar 
    \end{pmatrix} = 
    \begin{pmatrix}
        \vertbar &        & \vertbar \\
        \bm{v}_1 & \hdots & \bm{v}_n \\
        \vertbar &        & \vertbar 
    \end{pmatrix}
    \begin{pmatrix}
        \lambda_1 &        &           \\
                  & \ddots &           \\
                  &        & \lambda_n
        \end{pmatrix}
\end{align} or
\begin{align}
    \bm{AV} = \bm{V\Lambda}
\end{align}
Since the columns of $\bm{V}$ form a basis they are linear independent and one can invert $\bm{V}$. Thus if there is an
eigenbasis a matrix can be ”diagonalized”:
\begin{align}
    \bm{A} = \bm{V\Lambda V}^{-1} && \bm{\Lambda} = \bm{V}^{-1} \bm{\Lambda V}^{-1}
\end{align}

This form is often practical, for example it’s easy to compute
\begin{align}
    \bm{A}^p &=  \underbrace{\bm{V\Lambda V}^{-1} \bm{V\Lambda V}^{-1} \hdots \bm{V\Lambda V}^{-1}}_{p \ \text{times}} \\
             &=  \bm{V\Lambda}^p \bm{V}^{-1} = \bm{V} 
                \begin{pmatrix}
                    \lambda_1^p &        &             \\
                                & \ddots &             \\
                                &        & \lambda_n^p \\
                \end{pmatrix} \bm{V}^{-1}.
\end{align}


\section{Definiteness}
A is positive definite if for all: 
$\Vec{v} \neq 0 \hspace{0.5cm} V^TAV >0$
A is positive semi-definite if for all $\Vec{v} \ngtr 0 \hspace{0.5cm} V^TAV \geq 0$
positive definite $\Rightarrow$ %unique minimum\\
semi-definite $\Rightarrow$ %unique but not minimum


\section{Eigenvalues (put in at right place)}

\begin{equation*}
    \begin{array}{l}
      A =  
      \begin{pmatrix}
          a_{11} & a_{12} & a_{13} \\
          a_{21} & a_{22} & a_{23} \\
          a_{31} & a_{32} & a_{33}
      \end{pmatrix}
      \text{is symmetric if}\\
      A = A^T  \hspace{0.2cm} i.e. \hspace{0.2cm}if \hspace{0.2cm}a_{ij} = a_{ji}
    \end{array}
    \end{equation*}
    \vspace{10px}
    
    A real matrix can have complex eigenvalues
    \begin{equation*}
    \begin{array}{l}
      \begin{pmatrix}
          0 & 1 \\
          -1 & 0
      \end{pmatrix}
      \Vec{v} = \lambda\Vec{v}\\
      \\
      \begin{pmatrix}
          -\lambda & 1 \\
          -1 & -\lambda
      \end{pmatrix}
      \Vec{v} =0\\
      \\
      \det
      \begin{pmatrix}
          -\lambda & 1 \\
          -1 & -\lambda
      \end{pmatrix}
      =0\\
      \lambda^2+1=0
      \hspace{0.5cm}
      \lambda = \pm i
      \hspace{0.5cm}
      v_{1}=
      \begin{pmatrix}
          1 \\
          i
      \end{pmatrix}\\
      \vspace{0.5cm}
      \begin{pmatrix}
          -i & 1 \\
          -i & 1 
      \end{pmatrix}
      \begin{pmatrix}
          x_{1} \\
          y_{2}
      \end{pmatrix}
      =
      \begin{pmatrix}
          0 \\
          0 
      \end{pmatrix}\\
      -ix_{1}+y_{2} =0\\
      x_{1} =1\\
      y_{2} =i
    \end{array}
    \end{equation*}
    
    
    Note:
    \begin{equation*}
      \begin{array}{l}
        \begin{pmatrix}
            i & 1 \\
            -1 & i
        \end{pmatrix}
        \begin{pmatrix}
            x \\
            y
        \end{pmatrix}
        =
        \begin{pmatrix}
            0 \\
            0
        \end{pmatrix}\\
        x_{i}+y =0 \\
        x=1\\
        y=-i\\
        \Vec{v_{2}} =
        \begin{pmatrix}
          1 \\
          -i
        \end{pmatrix}\\
      \end{array}
    \end{equation*}
    
    Complex number: z=a+ib
    Complex conjugate is defined as $\bar{z}=a-ib$
    $z\bar{z} = (a+ib)(a-ib) = a^2 + b^2 \geq 0$
    
    \begin{equation*}
      \begin{array}{l}
      \text{Complex vector  } \Vec{v}=
        \begin{pmatrix}
            z_{1} \\
            z_{2} \\
            z_{3} \\
            \vdots \\
            z_{n}
        \end{pmatrix}
      \end{array}
    \end{equation*}
    
    \begin{center}
        $\bar{\Vec{v}} \cdot \Vec{v} \geq 0$ with equality iff $\Vec{v} = \Vec{0}$
    \end{center}
    \vspace{20px}
    
    Eigenvalues of symmetric real matrix are real. \\
    Suppose $\lambda$ is possibly complex eigenvalue of symmetric matrix A and eigenvalue $\Vec{\mathbf{V}} \neq 0$
    \vspace{10px}
    
    \begin{equation*}
      \begin{array}{l}
        \lambda \bar{\mathbf{V}}^T \mathbf{V} = \bar{\mathbf{V}}^T \lambda \mathbf{V} = \bar{\mathbf{V}}^T A\mathbf{V} = \bar{\mathbf{V}}^T A^T \mathbf{V} = (A\bar{\mathbf{V}})^T \mathbf{V} = (\bar{A}\bar{\mathbf{V}})^T \mathbf{V} = (\bar{\lambda}\bar{\mathbf{V}})^T = \bar{\lambda}\bar{\mathbf{V}}^T \mathbf{V}\\
        \Rightarrow \lambda = \bar{\lambda} \text{ since } \bar{\mathbf{V}}^T \mathbf{V} \neq 0\\
      \end{array}
    \end{equation*}
    
    
    \vspace{10px}
    
    \begin{equation*}
      \begin{array}{l}
        A\Vec{\mathbf{V}}=\lambda\Vec{\mathbf{V}}\\
        \sum a_{ij}\mathbf{V}_{j} = \lambda \mathbf{V}_{i}\\
        \overline{\sum a_{ij}\mathbf{V}_{j}} = \overline{\lambda\mathbf{V}_{i}}\\
        \sum_{i}\overline{a_{ij}}\overline{\mathbf{V}_{j}}=\bar{\lambda}\overline{\mathbf{V}_{i}}\\
        \Rightarrow \bar{A}\bar{v} = \bar{\lambda}\\
      \end{array}
    \end{equation*}
    
    \vspace{10px}
    
    Real symmetric matrices have an orthogonal eigenbasis. \\
    
    $\lambda_{1} \Vec{v_{1}}^T \Vec{v_{2}} = \Vec{v_{1}}^T A^T \Vec{V_{2}} = \Vec{v_{1}}^T A \Vec{V_{2}} = \Vec{v_{1}}^T \Vec{V_{2}} \lambda_{2}\\
    \Rightarrow \text{if } \lambda_{1} \neq \lambda_{2} \text{ and } \bar{v_{1}} \cdot v_{2} =0$ 
    and thus they are orthogonal.\\
    \vspace{10px}
    
    what if two eigenvectors to same eigenvalues?\\
    \begin{center}
        $A(a\Vec{v_{1}}+b\Vec{v_{2}})=aA\Vec{v_{1}}+bA\Vec{v_{2}}=a\lambda\Vec{v_{1}}+b\lambda\Vec{v_{1}}=\lambda(a\Vec{v_{1}}+b\Vec{v_{2}})$
    \end{center}
    
    
    Linear combination is also eigenvector.\\
    Use gram-schmidt to get orthogonal basis for this space.\\
    \vspace{10px}
    
    Each vector can be written in this eigenbasis as \\
    
    \begin{equation*}
      \begin{array}{l}
        b = \sum_{i=1}^{n}\Vec{\mathbf{V}}\Vec{\mathbf{V}}^T \Vec{b}\\ 
        A = A \sum_{i=1}^{n}\Vec{\mathbf{V}_{i}}\Vec{\mathbf{V}_{i}}^T = \sum_{i=1}^{n}A\Vec{\mathbf{V}_{i}}\Vec{\mathbf{V}_{i}}^T = \sum_{i=1}^{n}\Vec{\mathbf{V}_{i}}\lambda_{i}\Vec{\mathbf{V}_{i}}^T\\
        A\Vec{b} = \sum_{i=1}^{n}\Vec{\mathbf{V}_{i}}\lambda(\Vec{\mathbf{V}_{i}}^T \Vec{b})
      \end{array}
    \end{equation*}
    
    \vspace{5px}
    
    
    \begin{equation*}
      \begin{array}{l}
        A = \sum_{i=1}^{n}\Vec{\mathbf{V}_{i}}\lambda_{i}\Vec{\mathbf{\mathbf{V}}_{i}}^T\\
        A_{jk} = \sum_{i}^{}\mathbf{V}_{ij}\lambda_{i}\mathbf{V}_{ik}\\
        =\sum_{i}^{}\mathbf{V}_{ji}^T\lambda_{i}\mathbf{V}_{ik}\\
        =\sum_{i}^{}[\mathbf{V}^T \Lambda]_{ji} \mathbf{V}_{ik}\\
        =[\mathbf{V}^T \Lambda \mathbf{V}]_{jk}\\
        A = \mathbf{V}^T \Lambda \mathbf{V}
      \end{array}
    \end{equation*}
    
    $\mathbf{V}_{ij} \hdots $ j-th element of i-th eigenvector\\
    $\mathbf{V}_{ji}^T \lambda_{i} \hdots$ columne-wise multiplication of matrix  $\mathbf{V}^T$ with eigenvalues $\lambda_{i}$\\
    
    \begin{equation*}
      \begin{array}{l}
        \begin{pmatrix}
            \vdots & \vdots\\
            \Vec{\mathbf{V}_{1}} & \Vec{\mathbf{V}_{2}}\\
            \vdots & \vdots
        \end{pmatrix}
        \begin{pmatrix}
            \lambda_{1} & 0\\
            0 & \lambda_{2}
        \end{pmatrix}
        =
        \begin{pmatrix}
            \vdots & \vdots\\
            \lambda_{1}\Vec{\mathbf{V}_{1}} & \lambda_{2}\Vec{\mathbf{V}_{2}}\\
            \vdots & \vdots
        \end{pmatrix}
      \end{array}
    \end{equation*}

    \section{Singular value decomposition}
    The singular value decomposition is a useful way to factorize a matrix. 
    Any matrix $\bm{M} \in R^{m \x n}$ can be written as a product of three matrices
    \begin{align}
        \underset{(m \x n)}{\bm{M}}  = \underset{(m \x m)}{\bm{U}}\underset{(m \x n)}{\bm{\Sigma}} \underset{(n \x n)}{\bm{V}^T}\\,
    \end{align}
    
    where $\bm{U}$ and $\bm{V}$ are orthogonal 
    \begin{align}
        \bm{UU}^T = \bm{1}_{m \x m} && \bm{VV}^T = \bm{1}_{n \x n},
    \end{align}
    and $\bm{\Sigma}$ is diagonal.
    
    Without loss of generality we can assume that $m \geq n$. This would only result in empty columns instead of empty rows of $\bm{\Sigma}$.
    The shapes look approximately like this ($5 \x 3$) in the example.
    \begin{align}
        \begin{bmatrix}
        &        & \\
        &        & \\
        & \bm{M} & \\
        &        & \\
        &        & 
        \end{bmatrix}    = 
        \begin{bmatrix}
        & &        & & \\
        & &        & & \\
        & & \bm{U} & & \\
        & &        & & \\
        & &        & & 
        \end{bmatrix}_{m \x m}
        \begin{bmatrix}
        \sigma_1 &        &          \\
                 & \ddots &          \\
                 &        & \sigma_n \\
                 &        &          \\
                 &        &          
        \end{bmatrix}_{m \x n}
        \begin{bmatrix}
            &          & \\
            & \bm{V}^T & \\
            &          &  
        \end{bmatrix}_{n \x n}
    \end{align}
    
    \paragraph{Proof of existence}
    We will prove that such a representation always exists.
    We know that $\bm{M}^T \bm{M}$ has an orthogonal eigenbasis as it is symmetric, and non-negative eigenvalues as it is positive-definite,
    \begin{align}
        \bm{M}^T \bm{M} \bm{v}_i = \alpha_i \bm{v}_i, && \alpha_i \geq 0, &&  \bm{v}_i \cdot \bm{v}_j = \delta_{ij}.
    \end{align}
    Next we define vectors 
    \begin{align}
    \bm{u}_i = \frac{\bm{Mv}_i}{\alpha_i}    
    \end{align}
    and observe that they are also pairwise orthogonal
    \begin{align}
        \bm{u}_i \bm{u}_j = \frac{\bm{v}_i^T \bm{M}^T \bm{Mv}_i}{\sqrt{\alpha_i \alpha_j}} = \frac{\bm{v}_i^T \bm{v}_i \alpha_i}{{\sqrt{\alpha_i \alpha_j}}} = 
        \frac{\delta_{ij} \alpha_i}{{\sqrt{\alpha_i \alpha_j}}} = 
        \frac{\delta_{ij} \alpha_i}{{\sqrt{\alpha_i \alpha_i}}} = \delta_{ij}
    \end{align}
    
    We also get that the $\bm{u}_i$ are eigenvectors of $\bm{MM}^T$, 
    \begin{align}
         (\bm{M}\bm{M})^T \bm{u}_i
        =(\bm{M}\bm{M})^T \bm{Mv}_i \frac{1}{\sqrt{\alpha_i}} 
        =\bm{M}(\bm{M}^T \bm{M}) \bm{v}_i \frac{1}{\sqrt{\alpha_i}} 
        = \alpha_i \bm{M} \bm{v}_i  \frac{1}{\sqrt{\alpha_i}} 
        = \alpha_i \bm{u}_i
    \end{align}
    
    Next we assume that$\bm{M}^T \bm{M}$ has $l$  non-zero eigenvalues.
    We take the respective eigenvectors $\bm{v}_1, \hdots, \bm{v}_l$  and compute the corresponding $\bm{u}_i$ 
    simultaneously, while using $\sigma_i=\alpha_i^{-\frac{1}{2}}$,
    \begin{align}
        \begin{pmatrix}
        \vertbar &        & \vertbar \\
        \bm{u}_1 & \hdots & \bm{u}_l \\
        \vertbar &        & \vertbar 
        \end{pmatrix} =&
        \bm{M}
        \begin{pmatrix}
        \vertbar &        & \vertbar \\
        \bm{v}_1 & \hdots & \bm{v}_l \\
        \vertbar &        & \vertbar 
        \end{pmatrix}
        \begin{pmatrix}
        \sigma_1 &        &          \\
                 & \ddots &          \\ 
                 &        & \sigma_l
        \end{pmatrix} \\
        \begin{pmatrix}
        \vertbar &        & \vertbar \\
        \bm{u}_1 & \hdots & \bm{u}_l \\
        \vertbar &        & \vertbar 
        \end{pmatrix}
        \begin{pmatrix}
            \sigma_1 &        &          \\
                     & \ddots &          \\ 
                     &        & \sigma_l
            \end{pmatrix}
        =&
        \bm{M}
        \begin{pmatrix}
        \vertbar &        & \vertbar \\
        \bm{v}_1 & \hdots & \bm{v}_l \\
        \vertbar &        & \vertbar 
        \end{pmatrix} \\
        \begin{pmatrix}
        \vertbar &        & \vertbar \\
        \bm{u}_1 & \hdots & \bm{u}_l \\
        \vertbar &        & \vertbar 
        \end{pmatrix}
        \begin{pmatrix}
            \sigma_1 &        &          \\
                     & \ddots &          \\ 
                     &        & \sigma_l
        \end{pmatrix}
        \begin{pmatrix}
            \horzbar & \bm{v}_1 & \horzbar \\
                      & \vdots   &  \\
            \horzbar & \bm{v}_l &  \horzbar \\
        \end{pmatrix} 
        =&
        \bm{M}
        \begin{pmatrix}
        \vertbar &        & \vertbar \\
        \bm{v}_1 & \hdots & \bm{v}_l \\
        \vertbar &        & \vertbar 
        \end{pmatrix}
        \begin{pmatrix}
            \horzbar & \bm{v}_1 & \horzbar \\
                      & \vdots   &  \\
            \horzbar & \bm{v}_l &  \horzbar \\
        \end{pmatrix} 
    \end{align}
    
    To get to our desired factorization and isolate $\bm{M}$ we would like to get somehow get rid of the last to matrices on the right.
    However while $\bm{V}^T \bm{V} = \bm{1}_{l \x l}$, $\bm{V}^T \bm{V} = \bm{1}_{n \x n}$ does not hold in general, and we can't easily 
    cancel it out.
    We know however, that 
    \begin{align}
        \bm{v}_i^T \bm{M}^T \bm{M} \bm{v_i} =  \bm{v}_i^T \bm{v}_i \alpha_i \\
        \alpha_i = 0 \ \Rightarrow \ \bm{Mv}_i = 0
    \end{align}
    \allowdisplaybreaks
    
    This allows us to do the following, where the first equality holds because we use the full set of eigenvectors $n$ eigenvectors,
    \begin{align}
        &\bm{M} = \bm{M}
        \begin{pmatrix}
        \vertbar &        & \vertbar & \red{\vertbar    } & \red{        }& \red{ \vertbar }\\
        \bm{v}_1 & \hdots & \bm{v}_l & \red{\bm{v}_{l+1}} & \red{ \hdots }& \red{ \bm{v}_n }\\
        \vertbar &        & \vertbar & \red{\vertbar    } & \red{        }& \red{ \vertbar }
        \end{pmatrix}
        \begin{pmatrix}
            \horzbar & \bm{v}_1   & \horzbar \\
                     & \vdots     &          \\
            \horzbar & \bm{v}_l   & \horzbar \\
            \red{\horzbar} & \red{\bm{v}_l+1} & \red{\horzbar} \\
                           & \red{\vdots    } &                \\
            \red{\horzbar} & \red{\bm{v}_n  } & \red{\horzbar} \\
        \end{pmatrix} \\
        =&
        \begin{pmatrix}
        \vertbar       &        & \vertbar       & \red{\vertbar    }       & \red{        }& \red{ \vertbar }       \\
        \bm{M}\bm{v}_1 & \hdots & \bm{M}\bm{v}_l & \red{\bm{M}\bm{v}_{l+1}} & \red{ \hdots }& \red{\bm{M} \bm{v}_n } \\
        \vertbar       &        & \vertbar       & \red{\vertbar    }       & \red{        }& \red{ \vertbar }
        \end{pmatrix}
        \begin{pmatrix}
            \horzbar & \bm{v}_1   & \horzbar \\
                     & \vdots     &          \\
            \horzbar & \bm{v}_l   & \horzbar \\
            \red{\horzbar} & \red{\bm{v}_l+1} & \red{\horzbar} \\
                           & \red{\vdots    } &                \\
            \red{\horzbar} & \red{\bm{v}_n  } & \red{\horzbar} \\
        \end{pmatrix} \\
        =&
        \begin{pmatrix}
        \vertbar       &        & \vertbar       & \red{\vertbar} & \red{        } & \red{\vertbar} \\
        \bm{M}\bm{v}_1 & \hdots & \bm{M}\bm{v}_l & \red{\bm{0}  } & \red{ \hdots } & \red{\bm{0}}   \\
        \vertbar       &        & \vertbar       & \red{\vertbar} & \red{        } & \red{\vertbar}
        \end{pmatrix}
        \begin{pmatrix}
            \horzbar & \bm{v}_1   & \horzbar \\
                     & \vdots     &          \\
            \horzbar & \bm{v}_l   & \horzbar \\
            \red{\horzbar} & \red{\bm{v}_l+1} & \red{\horzbar} \\
                           & \red{\vdots    } &                \\
            \red{\horzbar} & \red{\bm{v}_n  } & \red{\horzbar} \\
        \end{pmatrix}\\
        =& \bm{M}\begin{pmatrix}
        \vertbar       &        & \vertbar       & \red{\vertbar} & \red{        } & \red{\vertbar} \\
        \bm{v}_1 & \hdots & \bm{v}_l & \red{\bm{0}  } & \red{ \hdots } & \red{\bm{0}}   \\
        \vertbar       &        & \vertbar       & \red{\vertbar} & \red{        } & \red{\vertbar}
        \end{pmatrix}
        \begin{pmatrix}
            \horzbar & \bm{v}_1   & \horzbar \\
                     & \vdots     &          \\
            \horzbar & \bm{v}_l   & \horzbar \\
            \red{\horzbar} & \red{\bm{v}_l+1} & \red{\horzbar} \\
                           & \red{\vdots    } &                \\
            \red{\horzbar} & \red{\bm{v}_n  } & \red{\horzbar} \\
        \end{pmatrix} \\
        =&
        \bm{M} \sum_{i=1}^l \bm{v}_i \bm{v}_i^T + \sum_{i=l+1}^n \bm{0} \ \bm{v}_i^T =
        \bm{M} \sum_{i=1}^l \bm{v}_i \bm{v}_i^T  \\
        =&     \bm{M}
        \begin{pmatrix}
        \vertbar &        & \vertbar \\
        \bm{v}_1 & \hdots & \bm{v}_l \\
        \vertbar &        & \vertbar 
        \end{pmatrix}
        \begin{pmatrix}
            \horzbar & \bm{v}_1 & \horzbar \\
                      & \vdots   &  \\
            \horzbar & \bm{v}_l &  \horzbar \\
        \end{pmatrix}.
    \end{align}
    This is just the rhs. of \eqref{?} and we get that 
    \begin{align}
        \bm{M} = 
        \underbrace{
            \begin{pmatrix}
                \vertbar &        & \vertbar \\
                \bm{u}_1 & \hdots & \bm{u}_l \\
                \vertbar &        & \vertbar 
            \end{pmatrix}
        }_{\bm{U}^{(1)}}
        \underbrace{
        \begin{pmatrix}
            \sigma_1 &        &          \\
                     & \ddots &          \\ 
                     &        & \sigma_l
        \end{pmatrix}
        }_{\bm{\Sigma}^{(1)}}
        \underbrace{
        \begin{pmatrix}
            \horzbar & \bm{v}_1 & \horzbar \\
                      & \vdots   &  \\
            \horzbar & \bm{v}_l &  \horzbar \\
        \end{pmatrix} 
        }_{\bm{V}^{(1)^T}}
    \end{align}
    This is the reduced singular value decomposition of matrix. 
    However, in many cases we would like the outer matrices to be 
    square and orthogonal. 
    We can achieve this via padding zeros in the diagonal matrix and 
    adding completing the rows/columns of the outer matrices to form full bases.
    
    We first look at the term 
    \begin{align}
        \bm{DV}^{(1)^T} = 
        \begin{pmatrix}
        \vertbar &        & \vertbar \\
        \bm{\sigma}_1 & \hdots & \bm{\sigma}_l \\
        \vertbar &        & \vertbar 
        \end{pmatrix}
        \begin{pmatrix}
            \horzbar & \bm{v}_1 & \horzbar \\
                      & \vdots   &  \\
            \horzbar & \bm{v}_l &  \horzbar \\
        \end{pmatrix}  = \sum_{i=1}^l \bm{\sigma_i} \bm{v_i}^T
    \end{align}
    Similar to above we can add rows to $\bm{V}^{(1)^T}$ so that it's rows form a
    complete set of orthonormal vectors. If we add zero columns to $\Sigma$ the overall product
    will not change as is evident from the outer product view of matrix multiplication.
    
    \begin{align}
        \bm{DV}^{(1)^T} =& \sum_{i=1}^l \bm{\sigma_i} \bm{v_i}^T                                      \\
                        =& \sum_{i=1}^l \bm{\sigma_i} \bm{v_i}^T + \sum_{i=l+1}^n \bm{0} \bm{v_i}^T   \\
                        =& \begin{pmatrix}
                                \vertbar      &        & \vertbar      & \vertbar          &        & \vertbar      \\
                                \bm{\sigma}_1 & \hdots & \bm{\sigma}_l & \bm{0}_{l+1} & \hdots & \bm{0}_n \\
                                \vertbar      &        & \vertbar      & \vertbar          &        & \vertbar 
                            \end{pmatrix}
                            \begin{pmatrix}
                                \horzbar & \bm{v}_1 & \horzbar \\
                                        & \vdots   &           \\
                                \horzbar & \bm{v}_l & \horzbar \\
                                \horzbar & \bm{v}_{l+1} & \horzbar \\
                                         & \vdots   &          \\
                                \horzbar & \bm{v}_n & \horzbar 
                            \end{pmatrix} \\
                        =& \bm{\Sigma}^{(2)} \bm{V}^T
    \end{align}
    We have added some columns to $\bm{\Sigma}^{(2)}$ which has a shape of $(l \x n)$. 
    $\bm{V}$ is of shape $(n \x n)$ and now contains a complete set of orthonormal vectors.
    Similar to the above we can add columns to $\bm{U}^{(1)}$ and compensate them via adding
    zero-rows to $\Sigma^{(2)}$.
    \begin{align}
        \bm{U}^{(1)} \bm{\Sigma}^{(2)} =&
        \begin{pmatrix}
        \vertbar &        & \vertbar \\
        \bm{u}_1 & \hdots & \bm{u}_l \\
        \vertbar &        & \vertbar 
        \end{pmatrix}
        \begin{pmatrix}
            \horzbar & \bm{\sigma}_1     & \horzbar \\
                     & \vdots            &          \\
            \horzbar & \bm{\sigma}_l     & \horzbar \\
        \end{pmatrix} \\
        =&\begin{pmatrix}
            \vertbar &        & \vertbar & \vertbar &        & \vertbar \\
            \bm{u}_1 & \hdots & \bm{u}_l & \bm{u}_{l+1} & \hdots & \bm{u}_m \\
            \vertbar &        & \vertbar & \vertbar &        & \vertbar \\
            \end{pmatrix}
            \begin{pmatrix}
                \horzbar & \bm{\sigma}_1     & \horzbar  \\
                         & \vdots            &           \\
                \horzbar & \bm{\sigma}_l     &  \horzbar \\
                \horzbar & \bm{0}_{l+1} & \horzbar  \\
                         & \vdots            &           \\
                \horzbar & \bm{0}_m     &  \horzbar \\
        \end{pmatrix} \\
        =& \bm{U \Sigma}
    \end{align}
    \allowdisplaybreaks
    
    This yields the SVD in its final form:
    \begin{align}
        \bm{M} = \bm{U \Sigma V^T} =
         \begin{pmatrix}
            \vertbar &        & \vertbar \\
            \bm{u}_1 & \hdots & \bm{u}_m \\
            \vertbar &        & \vertbar 
        \end{pmatrix}
        \overset{\quad  l \x l \quad \quad \quad\quad \blue{l \x (n-l)}}{
        \underset{\ \ \ \red{(m-l) \x l} \ \ \quad \green{(m-l) \x (n-l)}}{
        \begin{pmatrix}
                \sigma_1 &        &          & \blue{0      } & \blue{\hdots}  &  \blue{0}  \\
                         & \ddots &          & \blue{\vdots } & \blue{\ddots }  & \blue{\vdots }  \\
                         &        & \sigma_l & \blue{0      } & \blue{\hdots }  & \blue{0      }  \\
                \red{0     }   & \red{\hdots} & \red{0     } & \green{0     }  & \green{\hdots} & \green{0} \\
                \red{\vdots}   & \red{\ddots} & \red{\vdots} & \green{\vdots}  & \green{\ddots} & \green{\vdots}   \\
                \red{0     }   & \red{\hdots} & \red{0     } & \green{0     }  & \green{\hdots} & \green{0}
        \end{pmatrix}}}
        \begin{pmatrix}
            \horzbar & \bm{v}_1     & \horzbar \\
                     & \vdots            &          \\
            \horzbar & \bm{v}_n     & \horzbar \\
        \end{pmatrix},
    \end{align}
    The blue entries were added when we augmented the $\bm{v}_i$ and the red and green zeros when we 
    completed the $\bm{u}_i$.