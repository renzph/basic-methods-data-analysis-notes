\chapter{Principal component analysis}
\paragraph{Motivating example}
There is a park and we are given the locations of the trees in it
given two numbers.
It appears that they are approximately planted in a line with slight deviations.
However we would like to find a single direction which tells us the most about the location of tree.
In the figure it is evident that if we can get a very good estimate of where trees are,
by knowing how far to go along the red line. The (blue) distance orthogonal to it is of less importance.
If we encoded each point by it's red and blue distance we would find 
that the red ones vary a lot more than the blue ones.
Thus we will define the most important direction as the one along which the data has the highest variance.

\section{Variance maximization}
Given a vector $\bm{v}$ with unit length  $||\bm{v}||=1$ we can calculate the 
coordinate wrt. to it via a dot product,
\begin{equation}
    y_i = \bm{x}_i \cdot \bm{v}, 
\end{equation}
yielding a scalar for each data-point.
The variance of the values $\{y_1, \hdots, y_n\}$ is 
\begin{align}
    &\frac{1}{n} \sum_{i=1}^n y_i^2 - \bar{y}^2 \\
    =& \frac{1}{n} \sum_{i=1}^n (\bm{x}_i \bm{v})^2 - \left(\frac{1}{n} \sum_{i=1}^n \bm{x}_i\bm{v}\right)^2 =& \text{def. of $y_i$} \\
    =&\frac{1}{n} \sum_{i=1}^n \sum_{j=1}^m \sum_{k=1}^m x_{ij} v_j x_{ik} v_k - \bm{v}^T \bm{\mu} \bm{\mu}^T \bm{v} =&\text{write dot product as sum}\\
    =&\frac{1}{n}  \sum_{j=1}^m \sum_{k=1}^m v_j  \red{\sum_{i=1}^n x^T_{ji}  x_{ik}} v_k - \bm{v}^T \bm{\mu} \bm{\mu}^T \bm{v} =&\text{rearrange}\\
    =&\frac{1}{n}  \sum_{j=1}^m \sum_{k=1}^m v_j  \red{[\bm{X^T X}]_{ik}} v_k - \bm{v}^T \bm{\mu} \bm{\mu}^T \bm{v} =&\text{replace sum by m.m.}\\
    =&\frac{1}{n}  \sum_{j=1}^m \sum_{k=1}^m v_j  [\bm{X^T X}]_{ik} v_k - \bm{v}^T \bm{\mu} \bm{\mu}^T \bm{v} =&\text{replace sum by m.m.}\\
    =&\frac{1}{n}  \bm{v}^T \bm{X^T X} \bm{v}- \bm{v}^T \bm{\mu} \bm{\mu}^T \bm{v} =&\text{replace sums by m.m.}\\
    =&\bm{v}^T \left(\frac{1}{n}\bm{X^T X} - \bm{\mu} \bm{\mu}^T \right) \bm{v} =&\text{rearrange}\\
    =&\bm{v}^T \bm{C} \bm{v} & \text{def. of covariance}
\end{align}
Since $\bm{C}$ is real and symmetric we can express it using it's eigenvalues and eigenvectors,
\begin{equation}
    \bm{C} = \sum_{j=1}^m \lambda_j \bm{w}_j \bm{w}_j^T.
\end{equation}
Since $(w_1, \hdots, w_m)$ form an orthonormal basis we can also write $\bm{v}$ as linear combination of them,
\begin{equation}
    \bm{v} = \sum_{i=1}^m \alpha_i \bm{w}_i.
\end{equation}
Thus we shift problem of finding the entries of $\bm{v}$ to finding the $\alpha_i$ that maximize the variance.
We can write our objective as 
\begin{align}
    \bm{v^T} \bm{C v} &= \sum_{i=1}^m \alpha_i \bm{w}_i^T \sum_{j=1}^m \bm{w}_j\bm{w}_j^T \lambda_j \sum_{k=1}^m \alpha_k \bm{w}_k^T &\\
    &= \sum_{i=1}^m \sum_{j=1}^m \sum_{k=1}^m \alpha_{i} \alpha_{k} \lambda_{j} \red{\bm{w}_i^T \bm{w}_j} \blue{\bm{w}_j^T \bm{w}_k} &\\
    &= \sum_{i=1}^m \sum_{j=1}^m \sum_{k=1}^m \alpha_{i} \alpha_{k} \lambda_{j} \red{\delta_{ij}} \blue{\delta_{jk}}&\\
    &= \sum_{i=1}^m \sum_{j=1}^m  \alpha_{i} \alpha_{j} \lambda_{j} \delta_{ij} & \text{sum over $k$} \\ 
    &= \sum_{i=1}^m \alpha_{i} \alpha_{i} \lambda_{i}  & \text{sum over $j$} \\ 
    &= \sum_{i=1}^m \alpha_{i}^2 \lambda_{i}.
\end{align}

Since $||\bm{v}|| = 1$ we know that $\sum_{i=1}^m \alpha_i^2 = 1$. 
The above result gives us a weighted average which is maximal when $\alpha_1=1$ since $\lambda_1$ is (one of) the largest eigenvalue(s).
Thus the direction in which the variance of our data is maximal is just the direction of the eigenvector with the largest eigenvalue.

We now want to find the second most important direction. For this to make sense we need some additional requirement
for the new direction, otherwise we could get the same one again.
Since we now want to find a second vector we rename $\bm{v}$ to $\bm{v_1}$ and are looking for a 
vector $\bm{v_2}$ that maximizes $\bm{v}_2^T \bm{C} \bm{v}_2$, with the constraint that it has to be orthognal 
to $\bm{v}_1$, that is $\bm{v}_1 \cdot \bm{v}_2 = 0$.
If we write the second vector in terms of the eigenvectors of $\bm{C}$,
\begin{equation}
    \bm{v}_2 = \sum_{i=1}^m \beta_i \bm{w}_i,
\end{equation}
then 
\begin{equation}
    \bm{v}_1^T \bm{v}_2 = \sum_{i=1}^m \beta_i \bm{w}_i^T \bm{w}_1 = \sum_{i=1}^m \beta_i \delta_{1i} = \beta_1 = 0
\end{equation}
We also have that 
\begin{equation}
    \bm{v}_2^T \bm{C} \bm{v}_2 = \sum_{i=1}^m \lambda_i \beta_i^2,
\end{equation}
and similar to the reasoning above to maximize this under the orthogonality constraint
we have to set $\beta_2 = 1$ and thus $\bm{v}_2 = \bm{w}_2$.
\red{Problem: Why do we want the directions to be orthogonal. Somehow clear but I don't like the reasoning.
Is this easier to do with lossy compression?} 

\section{PCA as lossy compression}
Above we tried to find directions with maximal variance.
Next we want to look at PCA as lossy compression.
In general we could try to find a simple rule that allows us to reconstruct our data.
For example if we have bivariate data with the relation $y_i = x_i^2$ we only really need to store the $x_i$-values
for each datapoint as we can reconstruct the $y$-values from it.
But if we allow to general rules we run into the problem of overfitting.
Thus we will allow our compression method to use a linear transformations.

We now look at what happens when we 
leave some of the vectors out of the picture and only include $l$ components to encode our data.
\begin{equation}
    \bm{x_i} \approx \bm{\mu} + \sum_{j=1}^l \alpha_{ij} \bm{v}_j
\end{equation}

We can calculate the mean squared error of this approximation over the whole dataset.
\begin{align}
    \text{MSE} =& \frac{1}{n} \sum_{i=1}^n \left(\bm{x}_i - \bm{\mu} + \sum_{j=1}^l \alpha_{ij} \bm{v_j}\right)^2 \\
    =& \frac{1}{n} \sum_{i=1}^n \left(\bm{\mu} + \sum_{j=1}^m \alpha_{ij} \bm{v_j} - \bm{\mu} - \sum_{j=1}^l \alpha_{ij} \bm{v_j}\right)^2 \\
    =& \frac{1}{n} \sum_{i=1}^n \left(\sum_{j=l+1}^m \alpha_{ij} \bm{v_j} \right)^2 \\
    =& \frac{1}{n} \sum_{i=1}^n \sum_{j=l+1}^m \alpha_{ij} ^2 \\
    =& \frac{1}{n} \sum_{i=1}^n \sum_{j=l+1}^{m} \left( (\bm{x}_i - \bm{\mu})^T \bm{v}_j \right)^2 \\
    =& \frac{1}{n} \sum_{i=1}^n \sum_{j=l+1}^{m} \red{\left( (\bm{x}_i - \bm{\mu})^T \bm{v}_j \right)} \blue{\left( (\bm{x}_i - \bm{\mu})^T \bm{v}_j \right)} \\
    =& \frac{1}{n} \sum_{i=1}^n \sum_{j=l+1}^{m} \red{\sum_{a=1}^m (x_{ia} - \mu_a) v_{ja}} \blue{\sum_{b=1}^m (x_{ib} - \mu_b) v_{jb}}\\
    =& \sum_{j=l+1}^{m} \sum_{a=1}^m \sum_{b=1}^m v_{ja} \orange{\left(\frac{1}{n} \sum_{i=1}^n (x_{ia} - \mu_a) (x_{ib} - \mu_b) \right)} v_{jb}\\
    =& \sum_{j=l+1}^{m} \sum_{a=1}^m \sum_{b=1}^m v_{ja} \orange{C_{ab}} v_{jb}\\
    =& \sum_{j=l+1}^{m} \bm{v}_j^T \bm{C} \bm{v}_j\\
    =& \sum_{j=l+1}^{m} \bm{v}_j^T \lambda_j \bm{v}_j\\
    =& \sum_{j=l+1}^{m} \bm{v}_j^T \bm{v}_j \lambda_j \\
    =& \sum_{j=l+1}^{m} \lambda_j \\
\end{align}

This means that the mean squared error is given by the sum of the eigenvalues of eigenvectors not used in the approximation.
If we want to have low reconstruction error using only a few components we should use the ones with high eigenvalues.










